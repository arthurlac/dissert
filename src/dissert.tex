\documentclass[a4paper,10pt]{article}
\usepackage{listings}
\usepackage[utf8]{inputenc}
\usepackage[english]{babel}
\usepackage{amsfonts}
\usepackage{amsmath}
\usepackage{amssymb}
\usepackage{amsthm}
\usepackage{mathtools}
\usepackage{parskip}
\usepackage{stackrel}
\usepackage{url}

\newenvironment{spaced}[1]
  {\begin{minipage}[c]{\textwidth}\vspace{#1}}
  {\end{minipage}}

\theoremstyle{definition}
\newtheorem{definition}{Definition}[section]
\newtheorem{example}{Example}[section]
\newtheorem{theorem}{Thereom}[section]

\usepackage{tikz-cd}
\usetikzlibrary{arrows}
\tikzset{
    commutative diagrams/.cd,
    arrow style=tikz,
    diagrams={>=space}
}
\tikzcdset{
    arrow style=tikz,
    diagrams={>={Straight Barb[scale=0.8]}}
}

\newcommand\bind{\gg=}
\newcommand\name{it me}
\newcommand\uni{learns}

\setlength{\parindent}{0pt}
\setlength{\parskip}{1em}

\title{Algebraic Effects and Monads as Programming Techniques for Managing Effects}
\author{\name}
\date{May 2018}

\begin{document}
\maketitle
\begin{center}
BSc Computer Science
\end{center}

\pagebreak
This dissertation may be made available for
consultation within the University Library and
may be photocopied or lent to other libraries for the purposes of consultation.

\begin{spaced}{5em}
Signed:
\end{spaced}

\pagebreak
  \thispagestyle{empty}
  \begin{spaced}{4em}
    \begin{center}
      \LARGE\textbf{Algebraic Effects and Monads as Programming Techniques for Managing Effects}
    \end{center}
  \end{spaced}
  \begin{spaced}{3em}
    \begin{center}
      Submitted by: \name
    \end{center}
  \end{spaced}
  \begin{spaced}{5em}
    \section*{COPYRIGHT}

    Attention is drawn to the fact that copyright of this dissertation rests
    with its author. The Intellectual Property Rights of the products
    produced as part of the project belong to the author unless otherwise specified
    below, in accordance with the University of \uni's policy on intellectual property
   (see http://www.\uni.ac.uk/ordinances/22.pdf).

    This copy of the dissertation has been supplied on condition that anyone
    who consults it is understood to recognise that its copyright rests with its
    author and that no quotation from the dissertation and no information
    derived from it may be published without the prior written consent of
    the author.

    \section*{Declaration}
    This dissertation is submitted to the University of \uni\ in accordance
    with the requirements of the degree of Bachelor of Science in the
    Department of Computer Science. No portion of the work in this dissertation
    has been submitted in support of an application for any other degree
    or qualification of this or any other university or institution of learning.
    Except where specifically acknowledged, it is the work of the author.
  \end{spaced}

  \begin{spaced}{5em}
    Signed:
  \end{spaced}
\pagebreak

\section{Abstract}
Effects are essential and unavoidable
but a major source of unpredictability
and complexity, and thus bugs.
This document will discuss two tools to manage effects,
both of which originate from category theory,
monads and algebraic effects.
Many programmers are completely unaware of monads and algebraic effects,
some are unconvinced.
This paper will attempt to introduce them,
elucidate their relation,
and advocate for their effectiveness as programming techniques.
Finally I will present an abridged version
which presumes less familiarity with functional language conventions
with the same goal as the rest of this document;
the target audience being programmers.
In essence this document is part of a larger
proposition for category theory as a framework for better software construction;
the generalisation being that mathematics is a powerful tool to understand the world.

\pagebreak
\tableofcontents

\pagebreak
\section{What are effects?}
If we are going to assert that effects are necessary
and difficult to manage we should explain why this is.
Firstly, we shall explore what effects actually are.\\

\begin{example}
    A good place to start is to give several examples of effects
    \begin{itemize}
        \item Input/Output
        \item Mutable state
        \item Exceptions
        \item Non-Determinism and Probabilistic Non-Determinism
        \item Continuations
    \end{itemize}
\end{example}

Conceptually, programmers understand an \textit{effect} to be the semantics of any code
which is not entirely for deterministically calculating a result for a function.
We should note that side-effects are a subset of effects,
for example input/output and mutable states are side effects
however non-determinism is not.
The distinction being that a side effect
has an observable effect or interaction on the "outside world".\\

\begin{definition}
    A function $f$ is \textit{referentially transparent} if
    for any function $g$ which calls $f$,
    any call of $f$ can be replaced with the value returned by $f$
    without changing the value returned by $g$.

    An obvious example of non-referentially transparent,
    or \textit{referentially opaque}, function is
    \begin{equation}
        f(x) = \{x := x + 1;\ x\}
    \end{equation}
    This returns $x + 1$ but \textbf{also} changes the value of $x$ itself.
    Note that this implies that $x$ is not a value but a reference to a value;
    one can never change a value, it is only possible to change
    what one is referring to, (one can refer to a reference).

    This term originates from analytical philosophy
    and was introduced by Strachey.
    Referentially transparency implies that to find
    "the value of an expression which contains a sub-expression,
    the only thing we need to know about the sub-expression is its value.
    Any other features of the sub-expression,
    such as its internal structure,
    the number and nature of its components,
    the order in which they are evaluated or
    the colour of the ink in which they are written,
    are irrelevant to the value of the main expression"\cite{strachey2000fundamental}.\\
\end{definition}

\begin{definition}
    A \textit{pure} function is one which is both
    referentially transparent and which has the same semantics
    before and after the substitution of the function and its value
    has occurred. Pure functions are said to be free of effects.

    For example,
    a non-deterministic dice roll function is impure,
    it does not have the same semantics
    once a call of the function is replaced by a value.
\end{definition}

It should be noted that colloquially
the terms \textit{pure} and \textit{referentially transparent} are not fixed,
and are up to a point interchangeable when talking to some programmers.

Briefly consider \textit{coeffects} as a complication of this discussion.
Coeffects relate to the interaction of a program and its environment\cite{petricek2014coeffects},
(informally) consider it an inverse of a side-effect.
The program $ls$, which lists the files in the present working directory,
requires coeffects.
Haskell has an equivalent function $listDirectory$:
\begin{equation}
    listDirectory :: FilePath \rightarrow IO\ [FilePath]
\end{equation}

Obviously $ls$ is not referentially transparent,
it does not have any side-effects however.
It is definitely not a pure function.
Thus if we define pure functions to be those
which are free of effects then this
implies that coeffects are a subset of effects.
Coeffects, while interesting, are out of scope for this document,
but do serve to show the depth of the topic at hand.

Now we have established an intuition for what effects are,
we will explain why they lead to increased program complexity.

The first reason is that impure functions make it more
difficult to employ equational reasoning.
Purity is a powerful constraint to have on a function,
particularly when we want to prove properties of
our programs\cite{backus2007can}.

In a lazy language purity is even more consequential,
any pure function might be reordered or omitted.
If the function is not actually pure then this would
break the semantics of the program.

There exists a relation between referential transparency and
the (simply-typed) lambda calculus,
$\beta$-equality (or $\beta$-reduction) in particular.
\begin{equation}
\frac
{\Gamma, x : T \vdash t\prime : T\prime \quad \Gamma \vdash t : T}
{\Gamma \vdash (\lambda x.t\prime)t = t\prime[t/x]: T\prime }
\end{equation}
Consider $y, z : T$ where $y$ is a base value,
and $z$ is a function call ($z = \lambda x.t\prime$)
which returns the base value $y$,
\begin{align}
    z\,t &= (\lambda x.t\prime)t \\
         &= t\prime[t/x]
\end{align}
If $z$ is not pure then we do not have $\beta$-equality, i.e.
\begin{equation}
    z\,t \neq y
\end{equation}
$\beta$-equality is one of the fundamental properties
of the simply typed lambda calculus.
Abandoning purity means a departure from
the roots of functional programming.

The next reason why purity is desirable is that
a great deal of trouble can happen when the semantics
of a program is implicit rather than explicit.
Effects are usually not explicit in type systems,
this makes engineering and
maintaining larger programs
much more complicated.
Thus if untracked effects are a considerable source of errors then we should
wish to track them as best as possible.

Effect systems\cite{jouvelot1991algebraic} have been proposed as a solution,
"an effect system labels each function with its possible effects,
so a function type is now written $\tau \rightarrow \sigma \tau\prime$,
indicating a function that may have effects delimited by $\sigma$".
This method sees mainstream use in Java,
checked exceptions are an example of an effect system.
Wadler and Thiemann\cite{wadler2003marriage}
unify this approach with monads, which we will examine in this document.

The key benefit of type systems is that they allow us to identify erroneous programs
systematically,
we want to maximise the amount of erroneous programs the type-checker will not accept.
Being able to track and type effects is hugely beneficial.

Summarily, we can not avoid effects so
we have to ask in what ways we can manage effects?
What tools do we have as a programmer at our disposal?
We will examine two ideas, both originating from category theory,
which provide us tools for effectful programming.


\pagebreak
\section{Categorical Views of Effects}

\subsection{Overview}
In this section we will explore the theoretical underpinnings
of monads and algebraic effects.
In the 1960s category theorists constructed monads
Category theorists invented monads in the 1960s to concisely of universal algebra

the study of algebraic theories and their models or algebras. 

\cite{wadler1990comprehending}
\cite{hyland2007category}


From monads and Lawvere theories we have
the $\lambda_C$-calculus and algebraic effects.

Briefly,

From Moggi’s assignment to each set X of the set TX of values associated with a computational effect,
one has passed to the study of the operations associated with the computational effect
\cite{hyland2007category}

A key understanding is that operations generate monads
notions of computation determine monads
\cite{plotkin2001adequacy}

Plotkin and Power verified that and, later joined by Hyland,
saw that what were then regarded as computational effects could,
with one exception, fruitfully be seen as an instance and a development of universal algebra:
interactive input/output was soon recognised as an example involving no equations;
it was seen how to incorporate local state naturally; and the various ways of combining computational effects proved to be simple instances of combining Lawvere theories.
\cite{hyland2007category}


\subsection{Categories}
First we must introduce the concept of \textit{categories},

\begin{definition}
    A \textit{category} $C$ consists of
    \begin{itemize}
        \item A set $Ob\,C$, elements of which are called \textit{objects} of $C$.
        \item For each $X, Y \in Ob\,C$
            a set $Hom(X,Y)$ called the \textit{homset} from $X$ to $Y$.
        \item A \textit{composition function} $\circ$ such that
            \begin{equation}
                \circ : Hom(X,Y) \times Hom(Y,Z) \rightarrow Hom(X,Z)
            \end{equation}
        \item For all $X \in Ob\,C$ an element $Id_X$ of $Hom(X,\ X)$ such that
            \begin{equation}
            f \circ Id_X = Id_Y \circ f = f
            \end{equation}
        \item Composition is associative.  $f \circ (g \circ h) = (f \circ g) \circ h$
    \end{itemize}
\end{definition}

\par
An element $f$ of $Hom(X,Y)$ is called an \textit{arrow},
or a \textit{morphism}. The object $X$ is called the \textit{domain} of $f$ and $Y$ is
the \textit{codomain}.\\

\begin{example}
    The immediate example of a category is the category of sets.
    The objects in $C$ are (small) sets,
    a morphism from $X$ to $Y$ is a function $f : X \rightarrow Y$.
    The composition of Set is given by composition of functions,
    and the identity maps are given by the identity functions.
\end{example}

\subsection{Functors}
\begin{definition}
    A \textit{functor} $U : C \rightarrow D$ consists of
    \begin{itemize}
        \item A function $Ob\,U : Ob\,C \rightarrow Ob\,D$.
        \item A function $U : Hom_C(X,Y) \rightarrow Hom_D(UX, UY)$
            such that $U$ respects both composition and identity.
            I.e.
            \begin{equation}
                Uf \circ Ug = U(f \circ g)
            \end{equation}
            \begin{equation}
                U\,Id_X = Id_{UX}
            \end{equation}
    \end{itemize}
\end{definition}

\begin{definition}
    An \textit{endofunctor} is a functor $U : C \rightarrow C$;
    i.e. the domain and codomain of the functor are the same category $C$.
\end{definition}

\subsection{Natural transformations}
\begin{definition}
    Given categories $C$ and $D$,
    with functors $U, V : C \rightarrow D$
    a \textit{natural transformation} $\alpha : U \rightarrow V$
    consists of
    \begin{equation}
        \forall\ X \in Ob\,C\ \textrm{a map} \ \alpha_X : UX \rightarrow VX
    \end{equation}
    such that $\forall\ f : X \rightarrow Y$ the following commutes
    \begin{center}
        \begin{tikzcd}[sep=large]
            UX \rar{\alpha_X} \dar[swap]{Uf} & VX \dar{Vf} \\
            UY \rar{\alpha_Y}                & VY
        \end{tikzcd}
    \end{center}
    A natural transformation can be considered a morphism of functors.
\end{definition}

\subsection{Monads}
"Monads are among the most pervasive structures in category theory
and its applications: for example, they are central to the category-theoretic account of universal algebra"
\cite{mac2013categories}
\begin{definition}
    A \textit{monad} is defined as the triple
    \begin{itemize}
        \item An endofunctor $T : C \rightarrow C$
        \item A natural transformation $\eta : 1_{C} \rightarrow T$
        \item A natural transformation $\mu : T^2 \rightarrow T$
    \end{itemize}
    Such that the following diagrams commute
    \begin{center}
        \begin{tikzcd}[sep=large]
            T \rar{\eta_T} \drar[swap]{1_{C}} & T^2 \dar{\mu} & \lar[swap]{T\eta} \dlar{1_{C}} T \\
                                               & T            &
        \end{tikzcd}
        \quad
        \begin{tikzcd}[sep=large]
            T^3 \rar{T\mu} \dar[swap]{\mu_T} & T^2 \dar{\mu} \\
            T^2 \rar[swap]{\mu}                    & T
        \end{tikzcd}
    \end{center}
\end{definition}
%TODO
%which satisfies also an extra equalizing requirement: ηA: A → T A is an equalizer of ηTA and T(ηA), i.e. for any f:B → TA s.t. f;ηTA = f;T(ηA) there exists a unique m:B → A s.t. f = m;ηA3.

\subsection{Kleisli Category}
\begin{definition}
    Given a monad $(T,\eta,\mu)$ over a category $C$,
    the \textit{Kleisli category} $C_T$ consists of
    \begin{itemize}
        \item Objects of $C_T$ are objects from the underlying category $C$.
        \item $Hom_{C_T}(X,Y) = Hom_C (X,TY)$
        \item Identity morphisms in $C_T$ are $\eta$ in $C$
        \item Composition $f \circ g$ is $\mu(Tf)g$
    \end{itemize}
\end{definition}

Composition in $C_T$ can be described in more detail via the operator
\begin{equation}
    (-)^{*} : Hom(X, TY) \rightarrow Hom(TX, TY)
\end{equation}
where given a morphism $f: X \rightarrow TY$ we have
\begin{equation}
    f^{*} = \mu_{Y} \circ Tf
\end{equation}
Note that
\begin{equation}
    \eta_{X}^{*} = Id_{TX}
    \quad\textrm{and}\quad
    f^{*} \circ \eta _{X} = f
\end{equation}
Then we can define the \textit{Kleisli operator} $\gg$ where
\begin{equation}
    g \gg f = g^{*} \circ f
\end{equation}
\begin{equation}
    x
    \stackrel{f}{\rightarrow}     T y
    \stackrel{T g}{\rightarrow}   T T z
    \stackrel{\mu z}{\rightarrow} T z
\end{equation}
where $\gg$ has these axioms
\begin{equation}
    (f \gg g) \gg h \equiv f \gg (g \gg h)
\end{equation}
\begin{equation}
    \eta_Y \gg f \equiv f \equiv f \gg \eta_X
\end{equation}

\subsection{Computational Lambda Calculus}
Moggi \cite{moggi1989computational}
introduced categorical semantics for computation based on monads.
He extended the simply typed lambda-calculus to
the computational lambda-calculus, or $\lambda_c$-calculus,
which allows computations with effects such as
non-determinism, side effects, and continuations.
In this model "a program denotes a morphism from $A$
(the object of values of type $A$) to $TB$
(the object of computations of type B)"
for example
"partial computations (of type $B$) is the lifting $B + \{\bot\}$".
Another example would be the type \texttt{IO Int}
meaning interactive input/output computations of integers.
Whereas the simply typed lambda calculus is modeled by a cartesian closed category $C$;
a $\lambda_c$-model over a category $C$ with finite products is a strong monad $(T,\eta,\mu,t)$
together with a $T$-exponential for every pair $\langle A, B\rangle$ of objects in $C$
\cite{moggi1989computational}.

Here are two examples of the $\lambda_C$-calculus from
\cite{moggi1989computational}\cite{moggi1991notions}.
\vspace{5mm}

\begin{example}\label{lc1}
\end{example}
    Computations with side-effects:
    \begin{itemize}
        \item $T(-)$ is the functor $(-\times S)^S$, where $S$ is a nonempty set of stores.
            Intuitively a computation takes a store and returns a value together with the modified store.
        \item $\eta_A$ is the map $a \rightarrow (\lambda s:S.\langle a,s \rangle)$
        \item $\mu_A$ is the map $f \rightarrow (\lambda s:S.eval(fs))$,
            i.e. $\mu_A(f)$ is the computation that given a store $s$,
            first computes the pair computation-store $\langle f\prime,s\prime\rangle = fs$
            and then returns the pair value-store $\langle a,s\prime\prime\rangle = f\prime s\prime$.
    \end{itemize}
\vspace{5mm}

\begin{example}\label{lc2}
    Non-deterministic computations:
    \begin{itemize}
        \item $T(-)$ is the covariant powerset functor,
            i.e.  $T(A)$ = $P(A)$ and $T(f)(X)$ is the image of X along f
        \item $\eta_A$ is the singleton map $a \mapsto  \{a\}$
        \item $\mu_A(X)$ is the big union $\bigcup X$
    \end{itemize}
\end{example}

\subsection{Lawvere Theories}
The semantics of algebraic effects have a categorical basis as
"countable enriched Lawvere theories freely generated by $\dots$ operations and equations".
\cite{plotkin2004computational}
"In mathematical practice Lawvere theories arise
whenever one has a functor into a category with finite products
and one studies the natural transformations between finite products of the functor".
\cite{hyland2007category}
A more in-depth mathematical background of Lawvere theories is, regrettably,
beyond the scope of this document.

The essential semantic difference between monads and algebraic effects is that
where monads have a constructed object $TX$,
i.e. the computations of the type $X$,
algebraic effects have operations from which effects arise;
and thus $TX$ is derived not constructed.
Consider the monad $IO Int$ which is the object of computations
of the type $Int$;
from an algebraic effect perspective
one has the operation \texttt{print} which generates the effect

An example of the Lawvere Theory $L_{I/O}$ for input/output is \textit{generated} by the operations
\begin{equation}
read : I \rightarrow 1 \quad\textrm{and}\quad write : 1 \rightarrow O
\end{equation}
where $I$ is a countable set of inputs and $O$ of outputs.

\cite{plotkin2001adequacy}
interactive input/output is more directly modelled
by the Lawvere theory $L_{I/O}$ than by the corresponding monad
\begin{equation}
TX = \mu Y.(O \times Y + Y^I + X)
\end{equation}

An enriched Lawvere theory is generated by operations subject to equations.
Operations appear directly in describing programming languages,
but equations do not.
One of the equations for side-effects is
\begin{equation}
    updateloc,v (updateloc,v\prime (x)) = updateloc,v\prime (x)
\end{equation}
and the corresponding program assertion is
\begin{equation}
(l := x;let y be !l in M) = (l := x;M[x/y])
\end{equation}
This example is from \cite{plotkin2001adequacy}.

A potential advantage of algebraic effects over monads is that
the "
notion of countable enriched Lawvere theory
provides us with a natural way to describe
how computational effects may be combined.
"
\cite{plotkin2004computational}.
I.e. algebraic effects compose more naturally.


A further advantage may be that
"a class of theories that can be viewed as categories with a monad,
so that
any category with a monad is, up to equivalence (of categories with a monad), one of such theories.
Such a reformulation in terms of theories is more suitable for formal manipulation and more appealing to those unfamiliar with Category Theory."
\cite{moggi1991notions}


The Lawvere theory LE for exceptions is the free Lawvere theory generated by E operations $raise : 0 \rightarrow  1$,
where E is a set of exceptions.
In terms of operations and equations,
this corresponds to an E-indexed family of nullary operations with no equations.
%The monad on Set generated by LE is TE = − + E. More generally, the forgetful functor UL : Mod(LE,C) C induces the monad
%−+E on C, where E is the E-fold copower of 1, i.e.,  exists in C.
It is shown in [39]
that this countable Lawvere theory induces Moggi’s side-effects monad $(S \times \textrm{-})^S$ on
Set.
\cite{hyland2007category}


\pagebreak
\section{Monads in the Wild}
\subsection{Overview}
Typically a programmer will think of a monad as three function signatures
which correspond to the triple $(T,\eta,\mu)$;
of course much more informally.
Firstly \texttt{fmap} corresponds to the functor $T$,
then we have \texttt{return} and \texttt{join} which correspond to
the natural transformations $\eta$ and $\mu$ respectively.
We should note that the terms \texttt{fmap,return,join}
are not fixed, relatively often other names are used;
for example \texttt{map} instead of \texttt{fmap}
or \texttt{pure/unit} instead of \texttt{return},
however the semantics are exactly the same.

\begin{equation}
  \begin{split}
    fmap   &:: (x \rightarrow y) \rightarrow M x \rightarrow M y \\
    return &:: x \rightarrow M x                                 \\
    join   &:: M (M x) \rightarrow M x
  \end{split}
\end{equation}

To be a correct monad implementation it must be true that:
\begin{equation}
  \begin{split}
      \lambda f.\lambda x.return\ (fmap\ f\ x)
      &\equiv
      \lambda f.\lambda x.fmap\ (return \circ f)\ x)
      \\
      \lambda f.\lambda x.fmap\ f\ x
      &\equiv
      \lambda f.\lambda x.fmap\ f\ (fmap\ id\ x)
      \\
      \lambda f.\lambda x.join\ (join\ (fmap\ f\ x))
      &\equiv
      \lambda f.\lambda x.join(fmap\ f\ (join\ x))
  \end{split}
\end{equation}

Contrast these monad implementation laws with the diagrams
\begin{equation}
    \begin{tikzcd}[sep=large]
        T \rar{\eta_T} \drar[swap]{1_{C}} & T^2 \dar{\mu} & \lar[swap]{T\eta} \dlar{1_{C}} T \\
                                           & T            &
    \end{tikzcd}
    \quad
    \begin{tikzcd}[sep=large]
        T^3 \rar{T\mu} \dar[swap]{\mu_T} & T^2 \dar{\mu} \\
        T^2 \rar[swap]{\mu}                    & T
    \end{tikzcd}
\end{equation}

Do monad implementations need to follow these axioms?
There is no way for the compiler to check these rules so practically the answer is no,
however it is wise for monad implementations to obey them so that
we can chain computations in predictable and understandable ways.

%For programmers the most important of the triple is \texttt{fmap},
%this is because it is our primary way to interact with the monad.
%Typically we describe our desired result in terms of functions on the monad
%done one after the other.
However we can create a new function from this triple
\begin{align}
    bind &:: M x \rightarrow (x \rightarrow M y) \rightarrow M y \\
    bind &\equiv \lambda mx. \lambda f. join(fmap\ f\ mx)
\end{align}
Bind is incredibly useful and perhaps more prominent
to a programmer than the triple $(fmap,return,join)$.
This is because it allows us to chain computations
one after the other.
\begin{equation}
    x\
    \stackrel{f}{\rightarrow} My
    \stackrel{g}{\rightarrow} MMz
    \stackrel{join}{\rightarrow} Mz
\end{equation}
\begin{equation}
    \lambda g.
    \lambda f.
    \lambda x.
    join\ (fmap\ g\ (fmap\ f\ x))
\end{equation}
\begin{equation}
    (fmap\ f\ x) >>= g
\end{equation}

It is convention to use bind as the infix operator "$\bind$".
Consider that \texttt{join} can be expressed in terms of \texttt{bind} (with \texttt{id})
\begin{equation}
    join\ m \equiv m \bind id
\end{equation}
and furthermore so can \texttt{fmap} (with \texttt{return}).
\begin{equation}
    fmap\ f\ x \equiv (return\ x) \bind f
\end{equation}

%TODO bind relation to kleisli cat
Can we consider bind as a morphism between two objects in the Kleisli Category?

Because \texttt{bind} is often found with \texttt{(fmap,join,return)},
we can reformulate our axioms in terms of bind;
indeed the axioms are often reformulated this way because it is simpler.
\begin{align}
    return\ x \bind f &\equiv f x \\
    m \bind return &\equiv m \\
    (m \bind f) \bind g &\equiv m \bind (\lambda x.(f\ x \bind g))
\end{align}

We can further rephrase these axioms in terms
of the \textit{Kleisli composition operator} $\gg$ where
\begin{align}
    \gg &:: (x \rightarrow M y) \rightarrow (y \rightarrow M z) \rightarrow (x \rightarrow M z) \\
    f \gg g &\equiv \lambda x. (f\ x) >>= g
\end{align}
and we have
\begin{equation}
    (f \gg g) \gg h \equiv f \gg (g \gg h)
\end{equation}
\begin{equation}
    return \gg f \equiv f \equiv f \gg return
\end{equation}

\subsection{Monads for Structuring Programs}
In this section I will illustrate how effective monads are when used to structure programs.

It is well accepted that some algorithms become much more elegant
and simple when expressed in a functional paradigm

Firstly one should note that OCaml has non-nullable types
i.e; one will never see null or nil where one is expecting an
int or a binary tree or anything else.
Null values are always explicit.
The canonical representation of null is the None variant.
Monads can be used to succinctly and expressively structure computation with the option type.
%TODO Elucidate non-nullable types
%TODO Explain maybe
%Why is this?
%Why do we need monads to this?
%What do other languages do?

\begin{verbatim}
  type 'a option = Some 'a | None
  val return : 'a -> 'a option
  val join   : 'a option option -> 'a option
  val fmap   : 'a option -> ('a -> 'b) -> 'b option
\end{verbatim}

The code for these three functions is simple and fairly self-evident.
For example here is \texttt{join} for all values:
\begin{verbatim}
    join None            = None
    join (Some None)     = None
    join (Some (Some x)) = Some x
\end{verbatim}

However from this simple basis we can construct much more complicated programs which we
will be certain will never have a \textit{NullPointerException}.
Furthermore the type system will ensure 
It will refuse to compile nonsensical code which does not account for none
wherever it is possible to see none.
Such an assurance is invaluable in creating correct programs.
Here it's important to note that the effect is the \textit{NullPointerException};
we deal with it much more cleanly and effectively 100\% of the time using monads.

\begin{verbatim}
  let return a = Some a
  let join = function
      | Some (Some a) -> Some a
      | _ -> None
  let fmap a f = match a with
      | None -> None
      | Some x -> Some (f x)
\end{verbatim}

Consider this example code for searching a trie data structure.
Briefly, a trie is key value data structure;
where the key is a finite sequence of values
(typically a string which is equivalent to a sequence of characters).
Each node has an option value and a list of children.
The root of the trie represents the empty string.
Two auxiliary functions \texttt{find\_child} and \texttt{val\_extract}
are used in the search code; \texttt{val\_extract}
simply returns the first value in a pair,
find child searchs the list of children returning
the node with the matching character given;
that is only should that child exists.

\begin{verbatim}
  type ('k, 'v) t = Trie of 'v option * (('k * ('k, 'v) t) list)
  val find_child  : ('k, 'v) t -> 'k -> ('k, 'v) t option
  val val_extract : ('k, 'v) t -> 'v option
  val create      : 'k list -> 'v -> ('k, 'v) t
  val get         : ('k, 'v) t -> 'k list -> 'v option

  let create key data =
    let rec aux = function
      | []      -> Trie (Some data, [])
      | c :: cs -> Trie (None, [(c, aux cs)])
    in aux key
\end{verbatim}

Create is shown to just illustrate the data structure,
we simply iterate across the list creating a node in the trie for each character
until the list is exhausted at which point we insert the value.

Below is an example of a trie,
mapping list of characters to intergers,
at the root is the empty string,
then we have various key value pairs,
for example the key "tea" has the value 6.
Note that although "te" is part of the trie,
it has no value associated with that key.
Also note that keys not at leaf level can have associated
values, for example ("in", 4).
\begin{center}
    \begin{tikzpicture}[level distance=1.5cm,
  level 1/.style={sibling distance=3.5cm},
  level 2/.style={sibling distance=3cm},
  level 3/.style={sibling distance=2.5cm}]
  \node {$\varnothing$}
        child {node {$t$}
            child {node[text width=2cm,align=center] {$o$ \\ \textbf{"to" : 9}}}
            child {node {$e$}
                child {node[text width=2cm,align=center] {$a$ \\ \textbf{"tea" : 6}}}
                child {node[text width=2cm,align=center] {$d$ \\ \textbf{"ted" : 1}}}
                child {node[text width=2cm,align=center] {$n$ \\ \textbf{"ten" : 3}}}
            }
        }
    child {node {$a$}
            child {node[text width=2cm,align=center] {$t$ \\ \textbf{"at" : 5}}}
        }
    child {node {$i$}
            child {node[text width=4cm,align=center] {$n$ \\ \textbf{"in" : 4}}
                child {node[text width=2cm,align=center] {$n$ \\ \textbf{"inn" : 9}}}
                }
    };
\end{tikzpicture}
\end{center}

In the search code t is the trie, key is a list of characters over which
we iterate the search function. For each character we call find child on
the current node. Find child, if succesful, will return the next node
upon which we continue the search with the remaining characters. Once we
are at the last character we know to try and extract the value from the end
node. The key point here is that using bind allows us to succintly only code
for the happy case but deal with the error case at every step.

\begin{verbatim}
  let get t key =
    let rec search chars t =
        match chars with
        | []      -> val_extract t
        | c :: cs -> bind_search (find_child t c) cs
    and bind_search ot chars = ot >>= search chars
    in search key t
\end{verbatim}

The result type is similar to the option type, however we use an extra type
parameter for the unhappy case, essentially the result type encapsulates 
either an error or a correct computation result. The result monad
corresponds to a generalisation of the exception monad presented by Wadler \cite{wadler1995monads}.

\begin{verbatim}
  type ('a, 'e) result = Ok of 'a | Error of 'e
  val bind : ('a, 'e) result
          -> ('a -> ('b, 'e) result)
          -> ('b, 'e) result
\end{verbatim}

In this example we have a list assocation which is a list of pairs, in this case both
items in the pairs are strings.

\begin{verbatim}
  let list_assoc_to_job la =
    let find k = match List.Assoc.find la k with
      | None -> Error (Err.missing_key k)
      | Some v -> Ok v
    in
    find "name"   >>=                  (fun name   ->
    find "prog"   >>=                  (fun prog   ->
    find "args"   >>= parse_args   >>= (fun args   ->
    find "run_at" >>= parse_run_at >>= (fun run_at ->
      Ok (Job.create name prog args run_at ())
    ))))
\end{verbatim}

Bind is used to succintly short circuit a computation when a value can not be
correctly obtained. As such this allows the program to be structured neatly to return
an error with precise information for which key could not be found or which value could
not be parsed. The result monad is very similar to the option monad. It would be interesting
to examine whether algebraic effects can be used to structure a program in a similar manner.

Typically programming languages allow the elison of the anonymous function on the right hand side
of the bind to simply introduce the binding into the environment;
in this case OCaml ppx extension points are used in the form \textit{let\%bind}.
This is largely just a programmer convienence,
for comparison Haskell offers \textit{do notation} for the same end.

\begin{verbatim}
  let list_assoc_to_job la =
    let find k = match List.Assoc.find la k with
      | None -> Error (Err.missing_key k)
      | Some v -> Ok v
    in
    let%bind name   =  find "name"                     in
    let%bind prog   =  find "prog"                     in
    let%bind args   = (find "args"   >>= parse_args  ) in
    let%bind run_at = (find "run_at" >>= parse_run_at) in
    Ok (Job.create name prog args run_at ())
\end{verbatim}
Is equivalent to
\begin{verbatim}
  let list_assoc_to_job la =
    let find k = match List.Assoc.find la k with
      | None -> Error (Err.missing_key k)
      | Some v -> Ok v
    in
    find "name"   >>=                  (fun name   ->
    find "prog"   >>=                  (fun prog   ->
    find "args"   >>= parse_args   >>= (fun args   ->
    find "run_at" >>= parse_run_at >>= (fun run_at ->
      Ok (Job.create name prog args run_at ())
    ))))
\end{verbatim}

In Haskell \texttt{do} is simply syntactic sugar for \texttt{>>=} and \texttt{>>} where
\begin{equation}
    \texttt{>> :: M x -> M y -> M y}
\end{equation}
i.e. do first and ignore the result and then do the second.
Consider it equivalent to the semi colon operator in imperative languages.
\begin{align}
    \texttt{do}\ \{ x;\ \texttt{<stmts>} \}
    &\equiv x \texttt{ >> do \{<stmts>\}}
    \\
    \texttt{do}\ \{ v \leftarrow x\ \texttt{ <stmts>}\}
    &\equiv x \texttt{ >>= } \lambda v.\ \texttt{do}\ \{ \texttt{<stmts>} \}
    \\
    \texttt{do}\ \{\texttt{let }x = v\ \texttt{<stmts>}\}
    &\equiv (\lambda x.\ \texttt{do}\ \{ \texttt{<stmts>} \})v
\end{align}

In this example we simply and effectively deal with two types of errors.
Firstly a key being missing from our list
\begin{verbatim}
    let find k = match List.Assoc.find la k with
      | None -> Error (Err.missing_key k)
      | Some v -> Ok v
\end{verbatim}
Secondly a more complicated value which we have to parse;
the parsing of which can fail
\begin{verbatim}
    let%bind run_at = (find "run_at" >>= parse_run_at) in
\end{verbatim}
The key point here is when we have an error we want to fail
loudly and give as much information as possible about why we failed.
Furthermore, we don't want to take care of that in this function
we don't want to put that logic here
Monads allow us to sequence this error handling
and keep the necessary logic in a sensible re-usable place.

Finally consider the relation between
\texttt{(let\%bind,do)} and the Kleisli category of programs.
Consider how convienent it is for programmers to have a category
of programs where reliable error handling is built in by default.

\subsection{State Monad Example}
In this section we will illustrate how closely
the $\lambda_c$-calculus relates to actual code,
similar to what is in the Haskell standard library.
Furthermore we will show that from the relatively
simple triple $(fmap,return,join) + bind$ we can
construct code which has effects.

Firstly recall our example \ref{lc1} computations with
side-effects\cite{moggi1989computational}:
\begin{itemize}
    \item $T(-)$ is the functor $(-\times S)^S$, where $S$ is a nonempty set of stores.
        Intuitively a computation takes a store and returns a value together with the modified store.
    \item $\eta_A$ is the map $a \rightarrow (\lambda s.\langle a,s \rangle)$
    \item $\mu_A$ is the map $f \rightarrow (\lambda s.eval(fs))$,
        i.e. $\mu_A(f)$ is the computation that given a store $s$,
        first computes the pair computation-store $\langle f\prime,s\prime\rangle = f\ s$
        and then returns the pair value-store $\langle a,s\prime\prime\rangle = f\prime\ s\prime$.
\end{itemize}

We will examine how this maps to the Haskell state monad,
this implementation is similar to the one in the standard library
and is inspired by \cite{jones1995functional}.
We need to begin by declaring the type of the state monad,
it is effectively a closure over a function
which takes a state \texttt{s} and returns a value \texttt{a}.
In Moggi's work the function is anonymous however we name it
\texttt{runState}.
\begin{verbatim}
    newtype State s a = State {runState :: s -> (a, s)}
\end{verbatim}

Next \texttt{return},
consider $\eta_A$ the map $a \mapsto (\lambda s.\langle a,s \rangle)$,
this is a straightforward translation,
we just return a closure to an anonymous function
which takes a state and returns the pair \texttt{(a,s)}.
This anonymous function becomes \texttt{runState} for our
monad $State\ a$.
\begin{verbatim}
    return a = State (\ s -> (a,s))
\end{verbatim}

Moggi defines $\mu_A$ as the map $f \rightarrow (\lambda s.eval(fs))$,
we define this
\begin{verbatim}
    join :: (State s (State s a)) -> (State s a)
    join ssa = State (\s ->
        let a, s' = (runState ssa) s in
        (runState s') a
    )
    -- More idiomatically
    join ssa = State (\s -> uncurry runState (runState ssa s))
    -- where
    uncurry :: (a -> b -> c) -> ((a, b) -> c)
    uncurry f = \p ->  f (fst p) (snd p)
\end{verbatim}

Next fmap, which Moggi defines as
the functor $(-\times S)^S$,
where $S$ is a nonempty set of stores.
This corresponds to applying $f$ to $a$ and
a new state $s\prime$.
\begin{verbatim}
fmap :: (a -> b) -> State s a -> State s b
fmap f m = State $ \ s -> (f a, s')
    where (a, s') = runState m s
\end{verbatim}

At this point we should consider
that the Haskell implementation is remarkably
similar to the definition specified by Moggi,
of course it is not a completely direct translation
however the relation is obvious.
That said, we have not yet defined something "usable",
but we are close.

Finally we can define \texttt{bind} as
\begin{verbatim}
    m >>= k = State $ \s -> case runState m s of
                        (a, s') -> runState (k a) s'
    -- Or rephrased slightly
    m >>= f  = State (\ s ->
        let (x, s') = runState m s in
        runState (f x) s')
\end{verbatim}

Briefly, one should note that it appears many Haskellers
have a terrible fear of extraneous parenthesis,
as such \texttt{\$} can be read as "\texttt{(}"
and put the matching "\texttt{)}" somewhere appropriate.

Recall that we proposed:
\begin{equation}
    join\ m \equiv m \bind id
\end{equation}
Now is an ideal time to verify this in practise,
replace \texttt{k} for \texttt{id} and
observe that $id\ x \equiv x$ and we have
\begin{verbatim}
    m >>= id = State (\ s ->
        let (x, s') = runState m s in
        runState (id x) s')

    m >>= id = State (\ s ->
        let (x, s') = runState m s in
        runState x s')
\end{verbatim}
This is indeed equivalent to our previous
definition of \texttt{join}.

We have defined the triple,
one should ask how to actually do something useful;
we shall consider the example of a stack from \cite{lipovaca2011learn}.
\begin{verbatim}
type Stack = [Int]

pop :: Stack -> (Int, Stack)
pop = State $ \(x:xs) -> (x, xs)

push :: Int -> Stack -> ((), Stack)
push a = State $ \xs -> ((), a:xs)

stackManip = do
    push 3
    a <- pop
    pop

-- Which is equivalent to
stackManip stack =
    (push 3 >> (\ stack' ->
        stack' >>= a >> (\ stack'' ->
            pop stack''
        )
    )) stack

-- And equivalent to
stackManip stack = let
    ((), stack')  = push 3 stack
    (a , stack'') = pop stack'
    in pop stack''

ghci> stackManip [5,8,2,1]
(5,[8,2,1])
\end{verbatim}

\subsection{Monads for Imperative Programming}
Imperative programming is undeniably popular,
and much closer to how computers physically work than functional programming.
There is a bridge to be crossed between functional and imperative programming;
monads have been concretely shown to be an effective method of replicating
imperative style and retaining benefits of purity \cite{PeytonJones:1993}.
%TODO ins trans phrase here
These examples clearly illustrate how useful monads can be for structuring a program,
however the importance and necessity of monads is that for pure functional languages
to be actually useful we require effects. Monads give us means to achieve that.
In this section we will prove the inverse,
monads are good for imperative programming
and imperative programmers have monad-like features

In this section we will argue that monads are sometimes better
at managing effects than imperative language counterparts.

In the Go programming language one will frequently see this idiom
\begin{verbatim}
    result, err := myFunction()
    if err != nil {
        return nil, err
    }
\end{verbatim}

Go does not have exceptions,
so we must deal with errors inbetween function calls.
This is great in that it means the programmer can not
ignore errors, (or at least accidentally ignore an error).
Consider our error handling example;
in Go this might be much more verbose
and perhaps painfully explicit.

\begin{verbatim}
    name, err := findName(la)
    if err != nil {
        return nil, err
    }
    prog, err := findProg(la)
    if err != nil {
        return nil, err
    }
    etc...
    job.Create(name, prog, args, run_at)
\end{verbatim}

One can easily see the relation to the result monad which we have demonstrated.
Of course one would probably attempt to structure the program differently in Go;
however the OCaml implementation is undeniably straightforward
and error resistant, while providing useful error messages.
This reiterates the case that monads are effective tools
for constructing correct programs;
whilst still being concise and expressive!

Consider the Java programming language having checked exceptions,
i.e. every function that throws an exception must have a handler
at some point in the call stack.
We can consider this, in some ways, a slight improvement
over Go however one could argue we do not have as neat a solution
as monads give us.

We need a balance between explicit and implicit data flow
\cite{wadler1995monads}
"It is with regard to modularity that explicit data flow becomes both a blessing and a curse.
On the one hand, it is the ultimate in modularity.
All data in and all data out are rendered manifest and accessible, providing a maximum of flexibility. On the other hand, it is the nadir of modularity. The essence of an algorithm can become buried under the plumbing required to carry data from its point of creation to its point of use"
\cite{wadler1995monads}
A lot of the time bind does exactly what you want,

Indeed there it could be easy to argue that monads "force
one in effect to write a functional facsimile of an imperative program",
however there is a clear case that monads empower programmers
to retain benefits of functional programming and use aspects of imperative languages\cite{PeytonJones:1993},
%TODO rephrase
the proposition being that monads are akin to defining custom control flow operators.

\subsection{Monad Transformers}
The elephant in the room, when it comes to monads, is their composition.
Thus far we have shown that monads can be used for effects individually,
however when programming we often find that we need more than one effect
at a time.
For instance, a huge number of programs will need both state and IO,
hence we need means to combine these effects.

King and Wadler \cite{king1993combining}
describe how some monads method which operates on monads to yield a combined monad.
The idea is to add the functionality of one monad to another, underlying, monad.

They present an example of constructing a parser which needs both state
and the possibly to raise an exception,
returning either an error or a successful result.
We can achieve this desired combined monad in exactly two manners.
\begin{align}
    State\ Result\ a \rightarrow (Result\ a,\ State)\\
    Result\ State\ a \rightarrow Result\ (a,\ State)
\end{align}
These are not equivalent at all.
For $Result\ State\ a$ with any state obtained during the computation
is lost when the overall result is achieved.
This return type is either success or failure with no knowledge of the state.
Conversely in the first monad we have our state returned along with the result.
This is far more useful.
This example shows one of the key difficulties of combining monads,
the order in which we combine them matters greatly!
The analogy of composing functions is useful to explain why this is,
it is easy to see that for many $f$ and $g$ that
\begin{equation}
    g \circ f \not\equiv g \circ f
\end{equation}

In practise we can define a \textit{monad transformer}
to be a higher order monad which takes a monad
and produces a new combined monad with the effects of
monad given as the underlying monad.

The key function here is $lift$.
\begin{verbatim}
class MonadTrans t where
    -- | Lift a computation from the argument monad to the constructed monad.
    lift :: Monad m => m a -> t m a
\end{verbatim}

\begin{example}
    As a simple example we shall consider
    creating a transformer for the \texttt{Maybe} monad.
    %TODO Explain exactly what we get

    We shall look at this using code from \cite{haskellT}.
    Our desired result is the \texttt{MaybeT} monad transformer
    which accepts a monad $m$ and returns a monad $m\prime$
    with the effects of $m$ and maybe too.
    The naming convention for a monad transformer is to append "T".

    We begin by defining the type of \texttt{MaybeT}
    all we need is a monad \texttt{m} into which we will
    put a maybe value.
    \begin{verbatim}
    newtype MaybeT m a = MaybeT { runMaybeT :: m (Maybe a) }
    \end{verbatim}

    We can define \texttt{fmap} via \texttt{mapMaybeT}.
    \begin{verbatim}
    mapMaybeT :: (m (Maybe a) -> n (Maybe b)) -> MaybeT m a -> MaybeT n b
    mapMaybeT f = MaybeT . f . runMaybeT

    instance (Functor m) => Functor (MaybeT m) where
    fmap f = mapMaybeT (fmap (fmap f))
    \end{verbatim}
    Next we define return,
    which takes a value wraps it in \texttt{Just},
    applies the target monad return,
    and finally wraps it in \texttt{MaybeT}.
    \begin{verbatim}
    instance (Monad m) => Monad (MaybeT m) where
    return = MaybeT . return . Just
    \end{verbatim}

    Bind extracts the value,
    if it is nothing we wrap that via return.
    Otherwise we wrap the value applied to $f$.
    \begin{verbatim}
    x >>= f :: M (Maybe a) -> (a -> Maybe b) -> M (Maybe b)
    x >>= f = MaybeT $ do
        v <- runMaybeT x
        case v of
            Nothing -> return Nothing
            Just y  -> runMaybeT (f y)
    \end{verbatim}
    Finally we define the all important \texttt{lift} function,
    %TODO lift will allow us to...
    \begin{verbatim}
    instance MonadTrans MaybeT where
    lift m1 = MaybeT (m1 >>= \ x -> return (Just x))

    -- alternatively using
    liftM   :: (Monad m) => (a1 -> r) -> m a1 -> m r
    liftM f m1 = do { x1 <- m1; return (f x1) }

    -- resulting in
    lift = MaybeT . liftM Just
    \end{verbatim}
    %TODO Example maybe IO
\end{example}

%TODO Better phrasing
We have seen that we can combine monads,
using a monad transformer which takes one monad
and produces another;
naturally this leads us to potentially
having another monad transformer which we can input
a combined monad into.
Monad transformers involves
the creation of a \textit{stack} of monads,
this is a generic method of combining monads
such that we can combine the effects we need.

Monad transformers are effective,
and they will not be the largest issue in most programs
that use them;
however that is not to say they are perfect.
Firstly one has to note that, there is an efficiency tax,
"the State monad carries its state around in a closure.
Closures might be cheap in a Haskell implementation, but they're not free."
\cite{o2008real}
The next compliant is that we frequently see \textit{boilerplate code},
i.e. we need to explicitly lift useful functions through the stack
\cite{kammar2013handlers}.
A final criticism is that transformers exert inflexibility on implementations
due to requiring a static ordering of effects.
%TODO Cite static ordering

As a final note on monad transformers,
we should say that OCaml is impure and does not typically have monad transformers,
it doesn't need them.
This means OCaml can have combine monads, but it is a choice,
which frees us from a lot of the criticisms that monad transformers have.
Of course there remains the cost of unpredictable effects,
that said OCaml is typically pure by convention so one has to ask
what is the cost of complete purity?
%Why?
%How does this interact with modules composing?



\pagebreak
\section{Algebraic Effects}
Following from our discussion of Lawvere theories,
we will now explore their programming language progeny \textit{algebraic effects}.
The essence is we have \textit{operations} which generate effects,
and to actually interact with them as programming constructs
we need \textit{handlers}\cite{plotkin2009handlers};
which, informally, deconstruct effects.
This section will cover these two issues and their use.

For a casual understanding,
we can begin to explain algebraic effects by saying that
we have effects we can manipulate algebraically; that is to say,
the interpretation of an expression is
subject to an operator and its sub-expressions.
For instance, $a + b$ is interpreted as an operator
$+$ with its own semantics along with the value of its
two sub-expressions $a$ and $b$.
As our first example, consider the maybe monad where we have
\texttt{Nothing | Just x},
this monad is generated by
\begin{equation}
    raise :: 0 \rightarrow 1
\end{equation}
a nullary (of zero arguments) operator with no sub-expressions.

We can further our initial understanding with another view
relating algebraic effects to continuations,
or rather algebraic effects being given by continuations.
Whenever an operation is performed we have a continuation
passed to the handler;
which the handler uses to "implement" that effect.
This view also allows us an initial but intuitive
understanding as to why algebraic effects are incompatible with continuations;
it is not possible to distribute global control flow across several mechanisms.

However algebraic effects are not simply just continuations,
they are programmer defined effects.
To the programmer perhaps this is perhaps the most interesting implication,
we can now have programmer defined effects,
moreover potentially specifying multiple handlers for each effect.
Previously effects were completely implicit,
and this is a large reason why they can cause bugs,
now they are under direct and explicit control of the programmer.

Currently algebraic effects are a popular area of research,
as such we have seen a multitude of implementations in recent years.
They may well enter the mainstream (as mainstream as monads at least)
due to multicore OCaml being implemented in terms of algebraic effects.
Additionally, it is likely that imperative programmers
will find algebraic effects more intuitive than an equivalent monads.

\textbf{Some Algebraic Effect Implementations}
\begin{itemize}
    \item Eff: Matija Pretnar and Andrej Bauer,
        2015\cite{bauer2015programming}
    \item Frank: Sam Lindley and Conor McBride
        2016\cite{Lindley:2016vz}
    \item OCaml+effects: Stephen Dolan, Leo White, KC Sivaramakrishnan,
        2016\cite{ocamlplseff}
    \item Koka: Dan Leijen
        2016\cite{leijen:16}
\end{itemize}

\subsection{Overview}
\begin{example}
    Recall the examples of effects we gave at the beginning of this document,
    each of these effects has a natural operation which generates the effect.
    Consider
    \begin{itemize}
        \item Input/Output \textit{generated by} \texttt{read, write}
        \item Mutable state \textit{generated by} \texttt{set, get}
        \item Exceptions \textit{generated by} \texttt{raise}
        \item Non-Determinism \textit{generated by} \texttt{choose}
        \item Probabilistic Non-Determinism \textit{generated by} \texttt{choice}
    \end{itemize}
\end{example}

Notice that continuations are not mentioned here,
this is because, as discussed previously,
continuations are of a computationally different character
than these other effects
\cite{Plotkin:2002dw}\cite{hyland2007combining}.

Let us examine input and output as an algebraic effect in detail,
we will begin with the quintessential programming example; "Hello world!".
We will be working in OCaml+effects\cite{ocamlplseff} for this example,
using code from \cite{kceff}.\\

\begin{example}
    Consider this example of an IO effect,
    first we define an effect \texttt{IO},
    this is, in OCaml, similar to defining
    a new exception type.
    We then describe two constructors
    \texttt{Print} and \texttt{Read}
    of the effect \texttt{IO}.
    \begin{verbatim}
        effect IO =
          | Print : string -> unit
          | Read : string\end{verbatim}
    Now to actually have an effect we need to use the primitive \texttt{perform},
    we construct an effect via \texttt{(Print "Hello world!")}.
    \begin{verbatim}
        let print_hello () =
            perform (Print "Hello world!")\end{verbatim}
    Because OCaml is impure, the handler of this effect looks rather obtuse,
    however in a pure or lazy language this effect and handler would be necessary.
    Here it is just an indirection for the sake of an example.
    \begin{verbatim}
        let main () =
          try print_hello () with
          | effect (Print s) k -> printf s ^ "\n"; continue k\end{verbatim}
\end{example}

\subsection{Handlers}
Thus far we have only seen how to perform effects,
we need a mechanism by which to give these operations semantics.
Where operations are constructors of effects we also need a dual,
\textit{handlers} can be seen as deconstructors of effects\cite{Plotkin:2002dw}.

Handlers were introduced by Plotkin and Pretner\cite{Plotkin:2009dr},
the new handling construct generalises the exception-handling
construct of Benton and Kennedy\cite{benton2001exceptional}.
The key difference between typical exception handlers
and algebraic effect handlers is that
there is a continuation allowing for arbitrary effects to be expressed.

Handlers present one of the greatest benefits of algebraic effects;
"by separating effect signatures from their implementation,
algebraic effects provide a high degree of modularity"\cite{kammar2013handlers}.
Modularity is always important when engineering software
and allowing programmers to create an interface between
the expression of an effectful computation
and the concrete interpretation of their effects.

\subsection{Composing Algebraic Effects}
What has become the most evident deficiency of monads is
how difficult and tricky it is to compose them.
This is particularly unpleasant as
composability is often touted as one the most practical advantages of functional programming.
Fortunately composing algebraic effects is much less complicated than composing monads!
There exists an intuitive way to compose algebraic effects,
for example one doesn't need to consider order of composition as one does with monads.
This advantage stems from the theory in fact,
the "notion of countable enriched Lawvere theory
provides us with a natural way to describe
how computational effects may be combined
"\cite{plotkin2004computational}.

Another example of algebraic effects is binary non-determinism,
here we simulate a coin flip.\\
\begin{example}\begin{verbatim}

    effect choice = Choose : bool

    let coin_flip () =
        if (perform Choose) then
            Heads
        else
            Tails

    try coin_flip () with
    | effect Choose k -> continue k (Random.bool ())
    \end{verbatim}
\end{example}
Now let us tackle the question of how to actually compose
algebraic effects.\\

\begin{example}
    Recall our IO and non-determinism examples,
    we can combine them with ease!
\begin{verbatim}
    effect IO =
    | Print : string -> unit
    | Read : string

    effect choice =
    | Choose : bool

    effect int_state =
    | Get : int
    | Set : int -> unit

    let incr_maybe_twice () : int =
      perform (Set ((perform Get) + 1));
      if (perform Choose)
      then perform (Set ((perform Get) + 1))
      else perform (Print "incremented just once");
      perform Get
\end{verbatim}
    We simply need to add a handler for each effect we incorporate.
\begin{verbatim}
      match incr_maybe_twice () with
      | result -> result
      | effect (Put s') k -> (fun s -> continue k () s')
      | effect Get k -> (fun s -> continue k s s)
      | effect (Print s) k -> printf s; continue k
      | effect Choose k -> continue k (Random.bool ())


  let run f ~init =
    let comp =
      match f () with
      | x -> (fun s -> (s, x))
      | effect (Put s') k -> (fun s -> continue k () s')
      | effect Get k -> (fun s -> continue k s s)
    in comp init
\end{verbatim}
\end{example}

%TODO FIx EFFECTS EXANPLES!!!!!!!!!!!!
Of course having one global handler is neither elegant nor simple,
we are free to handle effects at any point in the call stack,
consider this example
\begin{verbatim}
    let incr_thrice () : int =
        let incr_aux () =
            perform Set ((perform Get) + 1);
            perform Set ((perform Get) + 1);
            perform Set ((perform Get) + 1)
        in
        match incr_aux () with
        | effect (Put s') k -> (fun s -> continue k () s')
        | effect Get k -> (fun s -> continue k s s)

    let incrx3_maybe () : int =
        let aux () =
            if (perform Choose)
            then incr_thrice ()
            else perform_get ()
        in
        match aux () with
        | effect Choose k -> continue k (Random.bool ())
\end{verbatim}

\subsection{Structuring Programs with Algebraic Effects}
As we demonstrated monads are rather useful for structuring programs,
for example,
allowing us to only consider the happy path while remaining assured
edge cases are handled;
it is natural to ask if programming with algebraic effects
has the same properties?

Recall our implementation of a trie with monads,
we will now explore whether an algebraic effect implementation
\begin{verbatim}
  let get t key =
    let rec search chars t =
        match chars with
        | []      -> val_extract t
        | c :: cs -> (find_child t c) >>= searh cs
    in search key t
\end{verbatim}

Contrast this with an algebraic effect implementation

\begin{verbatim}
  effect exception = Exn

  let get t key =
    let rec search chars t =
        match chars with
        | []      -> val_extract t
        | c :: cs -> search_children t c cs
    and search_children t c chars =
        match find_child t c with
        | None -> perform Exn
        | Some -> search chars
    in try search key t with
    | effect Exn _ -> None
    | value -> value
\end{verbatim}

We can immediately see that this implementation is more verbose,
which is not necessarily desirable;
however the mechanisms of how this function works
are more apparent.
%TODO Ugh pls struct w ae

\subsection{Conclusion}
There are two core benefits of algebraic effects.
Firstly, composition of algebraic effects is simple and pain free;
we will examine this later.
Secondly they allow a seperation between
"the expression of an effectful computation from its implementation"
\cite{dolan2015effective}.
%TODO Expl more here pls
This is a powerful tool of modular abstraction,
for very similar reasons as monads are powerful
for structuring programs.
Previously we saw that we could separate the happy path from our error handling,
it would appear that this separation is even more general with algebraic effects
and their handlers.
%TODO explain eff impl seper
%We can separate an effect from its implementation,

Finally we should note that
algebraic effects are particularly popular for systems programming
\cite{dolan2015effective, dolan2017concurrent, dolaneffectively};
especially in comparison to monads.
This is particularly due to performance implications,
implementing concurrency through algebraic effects
is comparable to the communicating sequential processes model
of Go in terms of performance \cite{Dolan:2017}.
This performance aspect is more striking
if we consider the efficiency tax of a monad
transformer stack\cite{o2008real}.

Algebraic effects may not be as elegant a tool as monads for structuring programs,
but it is undeniable they are far more composable and more efficient.
Thus they are more powerful tools for certain effects,
for example concurrency.

%lots of ppl say the monads when composed make effects coarse grained
%and this is ugly
%OCaml is impure but has monads, and soon effects, wen you need them?
%is this the most 'practical' approach?
algebraic effects are more scalable solution for engineering programs!
this maybe one of the reasons why they seem to be so popular in systems
programming

especially in comparison to monads.
This is particularly due to performance implications,
implementing concurrency through algebraic effects
is comparable to the communicating sequential processes model
of Go in terms of performance \cite{Dolan:2017}.
This performance aspect is more striking
if we consider the efficiency tax of a monad
transformer stack\cite{o2008real}.

Algebraic effects are currently an
immature programming language construct
however have immense potential because of 
their simplicity and flexibility.

Would suggest automatic handlers,
which would make the automatic behavuour of bind
available to algebraic effects
\begin{verbatim}
effect Result =
    | Return : 'a -> ()
    | Raise  : 'e -> ()
with default try
    | effect (Return v) -> Ok v
    | effect (Raise e) -> Err e
\end{verbatim}
Also note that this solves a common issue with
OCaml, there is no way to return early in a function
so one often has to hijack the current exception
system to emulate this behaviour.

Consider that we often see
\begin{verbatim}
effect Choose : bool
let choose () = perform Choose\end{verbatim}
or similarly
\begin{verbatim}
effect Put : t -> unit
let put v = perform (Put v)
\end{verbatim}
Perhaps it would be worthwhile to have this code
generated via ppx, for example
\begin{verbatim}
effect Put : t -> unit [@@deriving fun]
\end{verbatim}

These suggestions are motivated by the fact that
given a choice between a perfect but inconvenient tool
and a imperfect but convenient tool
a programmer will the vast majority of the time choose the latter.




\pagebreak
\section{Abridged version}

\subsection{Introduction}
Let's introduce bindables and resumable exceptions

\subsection{Bindables}
Roughly, a bindable is an interface
is something which takes a value
and then allows us to describe ways to chain functions
between bindables of the same type;
even if the things inside them are not of the same type

Bindables are defined by two functions
\begin{align}
    bind &:: B\ x \rightarrow (x \rightarrow List\ y) \rightarrow B\ y\\
    new &:: x \rightarrow B\ x
\end{align}
The way to read is that $::$ means "is of type",
the bind function takes two things,
firstly $B\ x$, which means a bindable of underlying type $x$.
Then $(x \rightarrow B\ y)$ is a function from type $x$,
to the type $B\ y$ which a bindable of type $y$
($y$ could be $x$ but doesn't have to be),
$x$ being the same type as our bindable.
The function returns a bindable of type $y$
You could think of $bind$ as an adapter,

The new function takes any type $x$ and returns
a function bindable of with the underlying type $x$.

Let's give a concrete example!
Foremost example of bindables are lists.

\begin{align}
    bind &:: List\ x \rightarrow (x \rightarrow List\ y) \rightarrow List\ y\\
    new &:: x \rightarrow List\ x
\end{align}

Now implementing $new$ is pretty easy.
\begin{verbatim}
    //Note we use 'unit' as return is a keyword
    const new = x => [x];
    console.log(new(1));
    // === Array [1]
\end{verbatim}

To define bind we gotta define an auxillary function
called join first, it takes a list of lists
and returns a list.
It basically just un-nests the list given.

\begin{verbatim}
    const arr2 = [[1, 2], [3, 4]];
    const append = (onto, val) => onto.concat(val);
    const join = x => x.reduce(append, []);
    console.log(join(arr2));
    // === Array [1, 2, 3, 4]
\end{verbatim}
Once we have done that, defining bind is pretty simple
\begin{verbatim}
    const bind = (x, f) => join(x.map(f));
    const double = x => [x,x];
    console.log(bind(arr1, double));
    // === Array [1, 1, 2, 2, 3, 3, 4, 4]
\end{verbatim}

Let's fix it to lists of integers just to make life ebeginasy
\begin{align}
    bind &:: List\ int \rightarrow (int \rightarrow List\ int) \rightarrow List\ int\\
    new &:: int \rightarrow List\ int
\end{align}

Or in OCaml
\begin{verbatim}
    let fmap f x =
        let rec aux l = match l with
            | [] -> []
            | hd :: tl -> (f hd) :: (aux tl)
        in aux x

    val fmap : ('a -> 'b) -> 'a list -> 'b list

    fmap (fun x -> x + 1) [1;2;3;4];;
    - : int list = [2; 3; 4; 5]
\end{verbatim}
Then we have
\begin{verbatim}
    let return x = [x];;

    val return : 'a -> 'a list

    return 1;;
    - : int list = [1]
\end{verbatim}
Then we have
\begin{verbatim}
    let rec join ll = match ll with
        | [] -> []
        | hd :: tl -> List.append hd (join tl)

    val join : 'a list list -> 'a list

    join [[1;2];[3;4]];;
    - : int list = [1; 2; 3; 4]
\end{verbatim}

Why are bindables so great?
They get rid of stuff like null-pointer exceptions!
They short circuit computations

\begin{verbatim}
    result, err := myFunction()
    if err != nil {
        return nil, err
    }
\end{verbatim}

We should make the distinction that
a monad in haskell $\neq$ a monad in OCaml $\neq$ a monad in category theory
At first this is obvious,
a mathematical construct is obviously different to a programming one,
however it is different between programming languages too,
we have different means to express structures in maths/Haskell/OCaml
thus any non-trivial structure is not really equivalent between these languages.
(maybe obvs but worth considering)
Do programming monads need to follow the same rules as category theory monads?
There is no way for the compiler to check these rules so practically the answer is no,
however it is wise for monad implementations to obey them so that
we can chain computations in predictable and understandable ways.

%For programmers the most important of the triple is \texttt{fmap},
%this is because it is our primary way to interact with the monad.
%Typically we describe our desired result in terms of functions on the monad
%done one after the other.

\subsection{Resumable Exceptions}
For programmers
we are not merely playing with delimited continuations (or resumable exceptions),

For a casual understanding,
we can begin to explain algebraic effects by saying that
we have effects we can manipulate algebraically; that is to say,
the interpretation of an expression is
subject to an operator and its sub-expressions.
For instance, $a + b$ is interpreted as an operator
$+$ with its own semantics along with the value of its
two sub-expressions $a$ and $b$.
As our first example, consider obtaining input from
\texttt{stdin} using the operation \texttt{read} where
\begin{equation}
    read :: \texttt{()} \rightarrow IO\ String
\end{equation}
as a nullary (of zero arguments) operator with no sub-expressions.

We can further our initial understanding with another view
relating algebraic effects to continuations,
or rather algebraic effects being given by continuations.
Whenever an operation is performed we have a continuation
passed to the handler;
which the handler uses to "implement" that effect.
This view also allows us an initial but intuitive
understanding of why algebraic effects are incompatible with continuations;
it is not possible to distribute global control flow across several mechanisms.


\begin{example}
    Consider this example of an IO effect,
    first we define an effect \texttt{IO},
    this is, in OCaml, similar to defining
    a new exception type.
    We then describe two constructors
    \texttt{Print} and \texttt{Read}
    of the effect \texttt{IO}.
    \begin{verbatim}
        effect IO =
          | Print : string -> unit
          | Read : string\end{verbatim}
    Now to actually have an effect we need to use the primitive \texttt{perform},
    we construct an effect via \texttt{(Print "Hello world!")}.
    \begin{verbatim}
        let print_hello =
            perform (Print "Hello world!")\end{verbatim}
    Because OCaml is impure, the handler of this effect looks rather obtuse,
    however in a pure or lazy language this effect and handler would be necessary.
    Here it is just an indirection for the sake of an example.
    \begin{verbatim}
        let () =
          try print_hello with
          | effect (Print s) k -> printf s; continue k\end{verbatim}
\end{example}
\subsection{Conclusion}

A lot of the time bind does exactly what you want,
handlers just force your hand in being explicit
It should be obvious that monads and algebraic effects
are not essential for programming,
or even correct and reliable programming.
However they are really helpful
One can perhaps suggest that pure languages are too dogmatic

"When writing a modular program to solve a problem, one first divides the problem into sub- problems, then solves the sub-problems and combines the solutions. The ways in which one can divide up the original problem depend directly on the ways in which one can glue solutions together. Therefore, to increase ones ability to modularise a problem conceptually, one must provide new kinds of glue in the programming language."
\cite{hughes1989functional}


\pagebreak
\section{Conclusion}
It is fundamentally more difficult to reason about code with effects
than without, however forgoing effects is forgoing usefulness itself.
This document presents a clear case that monads and algebraic effects
can be useful tools for engineering programs to be correct, maintainable, and understandable.

This project was motivated by my desire to understand both monads and algebraic effects,
as theoretical concepts and as tools.
At first I found algebraic effects, in particular, unapproachable,
the content in even the simplest papers were beyond my knowledge at the time.
I wanted to understand how these seemingly very disparate
things could be so closely closely related,
that they can express so much with such simple pieces.
Not only that, I wanted to explore
their direct usefulness and application in
every-day programming,
how I could use them to build programs,
and moreover why they were in some cases the right choice
and why they might be the wrong choice.
I feel in that end I have made considerable progress,
and look forward to continuing that progress.
Ideally I have produced something which has the information
necessary to build intuitive notions of the techniques discussed.

Above all I have found myself very curious at the idea that concepts
from mathematics can correspond to constructing elegant programs.

When one considers, however,
that category theory has been called the study
of "abstract nonsense"\cite{mac1997pnas},
it is no great surprise it can be so useful in software engineering,
when a great deal of software engineering is deciding what one
can make abstract while maintaining necessary structural properties.

\textbf{Acknowledgements}\\
Firstly I would like to thank my supervisor
for his advice, support, and contagious enthusiasm for category theory.
I would also like to thank for his thoughtful feedback
and expertise on JavaScript and engineering.

\pagebreak
\appendix

\section{\\Monadic Trie}
\begin{verbatim}
open Core

module Trie : sig
    type ('e, 'v) t

    val create   : key :'a list -> data : 'b -> ('a, 'b) t
    val add      : ('a, 'b) t -> key :'a list -> data : 'b -> ('a, 'b) t
    val get      : ('a, 'b) t -> key :'a list -> 'b option
    val is_entry : ('a, 'b) t -> key :'a list -> bool

end = struct
    type ('e, 'v) t = Trie of 'v option * (('e * ('e, 'v) t) list)

    let create ~key ~data =
        let rec aux = function
            | []      -> Trie (Some data, [])
            | c :: cs -> Trie (None, [(c, aux cs)])
        in aux key

    let val_extract      (Trie (v, _)) = v
    let children_extract (Trie (_, c)) = c

    let key_extract   (k, _) = k
    let child_extract (_, c) = c

    let find_child trie c =
        let cl = children_extract trie in
        let rec aux = function
            | [] -> None
            | t :: ts ->
                if c = (key_extract t)
                then Some (child_extract t)
                else aux ts
        in aux cl

    let get t ~key =
        let open Option.Monad_infix in
        let rec search chars t =
            match chars with
                | []      -> None
                | [c]     -> find_child t c >>= val_extract
                | c :: cs -> bind_search (find_child t c) cs
        (* Because find child is opt we want to short circuit sometimes *)
        and bind_search ot chars = ot >>= search chars
        in search key t

    let is_entry t ~key = get t ~key |> Option.is_some

    (* This function is find child but
     * also returns other children seperately. *)
    let seperate_child children key =
        let rec aux fwd bwd = match fwd with
            | [] -> None
            | n :: ns ->
                if (key_extract n) = key
                then Some ((child_extract n), ns @ bwd)
                else aux ns (n :: bwd)
        in aux children []

    let add t ~key ~data =
        let rec aux t chars =
            match chars with
                | [] -> (* Set the data *)
                    let Trie (_, children) = t in
                    Trie (Some data, children)
                | c :: cs -> (* Descend trie via current char c *)
                    let Trie (v, children) = t in
                    Trie (v, (descend children c cs))
        and descend children ch cs =
            match seperate_child children ch with
            (* Create new child *)
            | None -> (ch, (create ~key:cs ~data)) :: children
            (* Descend child *)
            | Some (child, others) -> (ch, (aux child cs)) :: others
        in aux t key
end
\end{verbatim}


\pagebreak
\bibliographystyle{unsrt}
\bibliography{dissert}

\end{document}
