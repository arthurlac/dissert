\documentclass[a4paper,10pt]{article}
\usepackage{listings}
\usepackage[utf8]{inputenc}
\usepackage[english]{babel}
\usepackage{amsfonts}
\usepackage{amsmath}
\usepackage{amssymb}
\usepackage{amsthm}
\usepackage{mathtools}
\usepackage{parskip}
\usepackage{stackrel}

\theoremstyle{definition}
\newtheorem{definition}{Definition}[section]
\newtheorem{example}{Example}[section]
\newtheorem{theorem}{Thereom}[section]

\usepackage{tikz-cd}
\usetikzlibrary{arrows}
\tikzset{
    commutative diagrams/.cd,
    arrow style=tikz,
    diagrams={>=space}
}
\tikzcdset{
    arrow style=tikz,
    diagrams={>={Straight Barb[scale=0.8]}}
}

\newcommand\bind{\gg=}

\setlength{\parindent}{0pt}
\setlength{\parskip}{1em}

\title{Algebraic Effects and Monads as Programming Techniques for Managing Effects}

\begin{document}
\section{Abstract}

KEY IDEAS:
Monads and Computation\\
Notions of Computation Determine Monads\\
Algebraic Effects\\
Purely Functional Data structures\\
Why Functional Programming Matters\\

Effects are tricky, and essential!
Programmers have long sought to make them more predictable;
this may come at the cost of complexity.

Two tools in dealing with effects, which both originate from category theory,
are monads and algebraic effects.

Most programmers are completely unaware of monads and algebraic effects.
Some are unconvinced.

This paper will attempt to introduce them and make case a for their use.

Elucidate the relation between monads and algebraic effects

their effectiveness as programming techniques.

AE and monads being state of the art????

I will present example code showing

Demonstrate monad-like programming techniques are already common in imperative languages

Explain where monad-like patterns arise in imperative programming and how monads are often a better alternative

THE KEY POINT IS MONADS ARE GREAT FOR STRUCTURING PROGRAMS TO BE understandable/reliable/correct

Finally I will present an abridged version which forgoes everything except the essentials
which presumes less familiarity with functional language conventions
with the goal of introducing and advocating for the concepts
the target audience being programmers
As such the success can be judged by how well it explains the use and benefits of both monads and algebraic effects.

Category theory as a framework for better software\\
as a generalisation of mathematics being a powerful tool to understand the world\\


\pagebreak
\tableofcontents

\pagebreak
\section{What are effects?}
If we are going to assert that effects are necessary
and difficult to manage we should explain why this is.
Firstly, we shall explore what effects actually are.\\

\begin{example}
    A good place to start is to give several example of effects
    \begin{itemize}
        \item Input/Output
        \item Mutable state
        \item Exceptions
        \item Non-Determinism and Probabilistic Non-Determinism
        \item Continuations
    \end{itemize}
\end{example}

Conceptually, programmers understand an \textit{effect} to be the semantics of any code
which is not entirely for deterministically calculating a result for a function.
We should note that side-effects are a subset of effects,
for example input/output and mutable states are side effects
however non-determinism is not.
The distinction being that a side effect
has an observable effect or interaction on the "outside world".\\

\begin{definition}
    A function $f$ is \textit{referentially transparent} if
    for any functon $g$ which calls $f$,
    any call of $f$ can be replaced with the value returned by $f$
    without changing the value returned by $g$.

    An obvious example of non-referentially transparent,
    or \textit{referentially opaque}, function is
    \begin{equation}
        f(x) = \{x := x + 1;\ x\}
    \end{equation}
    This returns $x + 1$ but \textbf{also} changes the value of $x$ itself.
    Note that this implies that $x$ is not a value but a reference to a value;
    one can never change a value, it is only possible to change
    what one is referring to. (One can refer to a reference).

    This term originates from analytical philosophy
    and was introduced by Strachey.
    Referentially transparency implies that to find
    "the value of an expression which contains a sub-expression,
    the only thing we need to know about the sub-expression is its value.
    Any other features of the sub-expression,
    such as its internal structure,
    the number and nature of its components,
    the order in which they are evaluated or
    the colour of the ink in which they are written,
    are irrelevant to the value of the main expression"\cite{strachey2000fundamental}.\\
\end{definition}

\begin{definition}
    A \textit{pure} function is one which is both
    referentially transparent and which has the same semantics
    before and after the substitution of the function and its value
    has occurred. Pure functions are said to be free of effects.

    For example,
    a non-deterministic dice roll function is impure,
    it does not have the same semantics
    once a call of the function is replaced by a value.
\end{definition}

It should be noted that colloquially
the terms \textit{pure} and \textit{referentially transparent} are not fixed,
and are up to a point interchangeable when talking to some programmers.

Briefly consider \textit{coeffects} as a complication of this discussion.
Coeffects relate to the interaction of a program and its environment\cite{petricek2014coeffects},
(informally) consider it an inverse of a side-effect.
The program $ls$, which lists the files in the present working directory,
requires coeffects.
Haskell has an equivalent function $listDirectory$:
\begin{equation}
    listDirectory :: FilePath \rightarrow IO\ [FilePath]
\end{equation}

Obviously $ls$ is not referentially transparent,
it does not have any side-effects however.
It is definitely not a pure function.
Thus if we define pure functions to be those
which are free of effects then this
implies that coeffects are a subset of effects.
Coeffects, while interesting, are out of scope for this document,
but do serve to show the depth of the topic at hand.

Now we have established an intuition for what effects are,
we will explain why they lead to increased program complexity.

The first reason is that impure functions make it more
difficult to employ equational reasoning.
Purity is a powerful constraint to have on a function,
particularly when we want to prove properties of
our programs\cite{backus2007can}.

In a lazy language purity is even more consequential,
any pure function might be reordered or omitted.
If the function is not actually pure then this would
break the semantics of the program.

There exists a relation between referential transparency and
the (simply-typed) lambda calculus,
$\beta$-equality (or $\beta$-reduction) in particular.
\begin{equation}
\frac
{\Gamma, x : T \vdash t\prime : T\prime \quad \Gamma \vdash t : T}
{\Gamma \vdash (\lambda x.t\prime)t = t\prime[t/x]: T\prime }
\end{equation}
Consider $y, z : T$ where $y$ is a base value,
and $z$ is a function call ($z = \lambda x.t\prime$)
which returns the base value $y$,
\begin{align}
    z\,t &= (\lambda x.t\prime)t \\
         &= t\prime[t/x]
\end{align}
If $z$ is not pure then we do not have $\beta$-equality, i.e.
\begin{equation}
    z\,t \neq y
\end{equation}
$\beta$-equality is one of the fundamental properties
of the simply typed lambda calculus.
Abandoning purity means a departure from
the roots of functional programming.

The next reason why purity is desirable is that
a great deal of trouble can happen when the semantics
of a program is implicit rather than explicit.
Effects are usually not explicit in type systems,
this makes engineering and
maintaining larger programs
much more complicated.
Thus if untracked effects are a considerable source of errors then we should
wish to track them as best as possible.

Effect systems\cite{jouvelot1991algebraic} have been proposed as a solution,
"an effect system labels each function with its possible effects,
so a function type is now written $\tau \rightarrow \sigma \tau\prime$,
indicating a function that may have effects delimited by $\sigma$".
This method sees mainstream use in Java,
checked exceptions are an example of an effect system.
Wadler and Thiemann\cite{wadler2003marriage}
unify this approach with monads, which we will examine in this document.

The key benefit of type systems is that they allow us to identify erroneous programs
systematically,
we want to maximise the amount of erroneous programs the type-checker will not accept.
Being able to track and type effects is hugely beneficial.

Summarily, we can can not avoid effects so
we have to ask in what ways can we manage effects?
What tools do we have as a programmer at our disposal?
We will examine two ideas, both originating from category theory,
which provide us tools for effectful programming.


\pagebreak
\section{Categorical Views of Effects}

\subsection{Overview}
In this section we will explore the theoretical underpinnings of effects,
the two key theoretical concepts are monads and Lawvere theories.
For a prima facie understanding of we can say that
they share a mathematical relation in understanding and expressing aspects of universal algebra.
Where abstract algebra has specific algebraic structures, such as groups and rings,
universal algebra examines these structures in a general sense.
Not only do they share a history, finitary monads and Lawvere theories
"describe equivalent categorical encodings of universal algebra"\cite{riehl}.
%TODO
These ideas, while fascinating and perplexing,
are not central to this document.
% and is added for perspective.
% They have been discussed to enrich the perspectives discussed

Moggi in 1989 \cite{moggi1989computational}
first described the relation between monads and computational effects,
introducing the $\lambda_c$-calculus.
Serendipitously the Haskell programming language\cite{hudak1992report}
was being formulated in this same period of time.
As both a pure and lazy language
it could not have effects in the manner programmers were accustomed to thus far;
Moggi's $\lambda_c$-calculus was brought forth by Wadler\cite{wadler1990}
as an elegant solution to this problem.

Plotkin and Power \cite{Plotkin:2001jr}
introduced Lawvere theories to the discussion of effects
in the early 2000s,
contributing that monads were not effects themselves but rather were determined by effects;
effects arising from operations; proposing "computational effects as being realised by
families of operations, with a monad being generated by their equational theory"\cite{Plotkin:2002dw}.

\subsection{Categories}
As one can guess from the name \textit{category theory},
categories are at the heart of the subject.
From this definition we will eventually arrive at monads.
A computer scientist can begin conceptually
understanding objects in a category
as an analogy to types
and morphisms as functions.\\

\begin{definition}
    A \textit{category} $C$ consists of
    \begin{itemize}
        \item A set $Ob\,C$, elements of which are called \textit{objects} of $C$.
        \item For each $X, Y \in Ob\,C$
            a set $Hom(X,Y)$ called the \textit{homset} from $X$ to $Y$.
        \item A \textit{composition function} $\circ$ such that
            \begin{equation}
                \circ : Hom(X,Y) \times Hom(Y,Z) \rightarrow Hom(X,Z)
            \end{equation}
        \item For all $X \in Ob\,C$ an element $Id_X$ of $Hom(X,\ X)$ such that
            \begin{equation}
            f \circ Id_X = Id_Y \circ f = f
            \end{equation}
        \item Composition is associative.  $f \circ (g \circ h) = (f \circ g) \circ h$
    \end{itemize}
\end{definition}

\par
An element $f$ of $Hom(X,Y)$ is called an \textit{arrow},
or a \textit{morphism}. The object $X$ is called the \textit{domain} of $f$ and $Y$ is
the \textit{codomain}.\\

\begin{example}
    The immediate example of a category is the category of sets.
    The objects in $C$ are (small) sets,
    a morphism from $X$ to $Y$ is a function $f : X \rightarrow Y$.
    The composition of Set is given by composition of functions,
    and the identity maps are given by the identity functions.
\end{example}

\subsection{Functors}
Functors are the next step in our journey,
and we can consider them a morphism between categories.\\

\begin{definition}
    A \textit{functor} $U : C \rightarrow D$ consists of
    \begin{itemize}
        \item A function $Ob\,U : Ob\,C \rightarrow Ob\,D$.
        \item A function $U : Hom_C(X,Y) \rightarrow Hom_D(UX, UY)$
            such that $U$ respects both composition and identity.
            I.e.
            \begin{equation}
                Uf \circ Ug = U(f \circ g)
            \end{equation}
            \begin{equation}
                U\,Id_X = Id_{UX}
            \end{equation}
    \end{itemize}
\end{definition}

\begin{definition}
    An \textit{endofunctor} is a functor $U : C \rightarrow C$;
    i.e. the domain and codomain of the functor are the same category $C$.
\end{definition}

\subsection{Natural transformations}
Just as we considered functors a morphism between categories,
we can in turn consider a natural transformation a morphism of functors.\\

\begin{definition}
    Given categories $C$ and $D$,
    with functors $U, V : C \rightarrow D$
    a \textit{natural transformation} $\alpha : U \rightarrow V$
    consists of
    \begin{equation}
        \forall\ X \in Ob\,C\ \textrm{a map} \ \alpha_X : UX \rightarrow VX
    \end{equation}
    such that $\forall\ f : X \rightarrow Y$ the following commutes
    \begin{center}
        \begin{tikzcd}[sep=large]
            UX \rar{\alpha_X} \dar[swap]{Uf} & VX \dar{Vf} \\
            UY \rar{\alpha_Y}                & VY
        \end{tikzcd}
    \end{center}
\end{definition}

\subsection{Monads}
Building upon our previous definitions we can define a monad.
Monads are a ubiquitous structure in category theory;
being central to the
"category-theoretic account of universal algebra"\cite{mac2013categories}.\\

\begin{definition}
    A \textit{monad} is defined as the triple
    \begin{itemize}
        \item An endofunctor $T : C \rightarrow C$
        \item A natural transformation $\eta : 1_{C} \rightarrow T$
        \item A natural transformation $\mu : T^2 \rightarrow T$
    \end{itemize}
    Such that the following diagrams commute
    \begin{center}
        \begin{tikzcd}[sep=large]
            T \rar{\eta_T} \drar[swap]{1_{C}} & T^2 \dar{\mu} & \lar[swap]{T\eta} \dlar{1_{C}} T \\
                                               & T            &
        \end{tikzcd}
        \quad
        \begin{tikzcd}[sep=large]
            T^3 \rar{T\mu} \dar[swap]{\mu_T} & T^2 \dar{\mu} \\
            T^2 \rar[swap]{\mu}                    & T
        \end{tikzcd}
    \end{center}
\end{definition}

\subsection{Kleisli Category}
\begin{definition}
    Given a monad $(T,\eta,\mu)$ over a category $C$,
    the \textit{Kleisli category} $C_T$ consists of
    \begin{itemize}
        \item Objects of $C_T$ are objects from the underlying category $C$.
        \item $Hom_{C_T}(X,Y) = Hom_C (X,TY)$
        \item Identity morphisms in $C_T$ are $\eta$ in $C$
        \item Composition $f \circ g$ is $\mu(Tf)g$
    \end{itemize}
\end{definition}

Composition in $C_T$ can be described in more detail via the operator
\begin{equation}
    (-)^{*} : Hom(X, TY) \rightarrow Hom(TX, TY)
\end{equation}
where given a morphism $f: X \rightarrow TY$ we have
\begin{equation}
    f^{*} = \mu_{Y} \circ Tf
\end{equation}
Note that
\begin{equation}
    \eta_{X}^{*} = Id_{TX}
    \quad\textrm{and}\quad
    f^{*} \circ \eta _{X} = f
\end{equation}
Then we can define the \textit{Kleisli operator} $\gg$ where
\begin{equation}
    g \gg f = g^{*} \circ f
\end{equation}
\begin{equation}
    x
    \stackrel{f}{\rightarrow}     T y
    \stackrel{T g}{\rightarrow}   T T z
    \stackrel{\mu z}{\rightarrow} T z
\end{equation}
where $\gg$ has these axioms
\begin{equation}
    (f \gg g) \gg h \equiv f \gg (g \gg h)
\end{equation}
\begin{equation}
    \eta_Y \gg f \equiv f \equiv f \gg \eta_X
\end{equation}

\subsection{Computational Lambda Calculus}
Moggi \cite{moggi1989computational}
introduced categorical semantics for computation based on monads.
He extended the simply typed lambda-calculus to
the computational lambda-calculus, or $\lambda_c$-calculus,
which allows computations with effects such as
non-determinism, side effects, and continuations.
In this model "a program denotes a morphism from $A$
(the object of values of type $A$) to $TB$
(the object of computations of type B)"
for example
"partial computations (of type $B$) is the lifting $B + \{\bot\}$".
A concrete example would be the type \texttt{IO Int}
meaning interactive input/output computations returning integers.
Whereas the simply typed lambda calculus is modeled by a cartesian closed category $C$;
a $\lambda_c$-model over a category $C$ with finite products is a strong monad $(T,\eta,\mu,t)$
together with a $T$-exponential for every pair $\langle A, B\rangle$ of objects in $C$
\cite{moggi1989computational}.

Here are two examples of the $\lambda_c$-calculus from
\cite{moggi1989computational}\cite{moggi1991notions}.
\vspace{5mm}

\begin{example}\label{lc1}
\end{example}
    Computations with side-effects:
    \begin{itemize}
        \item $T(-)$ is the functor $(-\times S)^S$, where $S$ is a nonempty set of stores.
            Intuitively a computation takes a store and returns a value together with the modified store.
        \item $\eta_A$ is the map $a \rightarrow (\lambda s:S.\langle a,s \rangle)$
        \item $\mu_A$ is the map $f \rightarrow (\lambda s:S.eval(fs))$,
            i.e. $\mu_A(f)$ is the computation that given a store $s$,
            first computes the pair computation-store $\langle f\prime,s\prime\rangle = fs$
            and then returns the pair value-store $\langle a,s\prime\prime\rangle = f\prime s\prime$.
    \end{itemize}
\vspace{5mm}

\begin{example}
    Non-deterministic computations:
    \begin{itemize}
        \item $T(-)$ is the covariant powerset functor,
            i.e.  $T(A)$ = $P(A)$ and $T(f)(X)$ is the image of X along f
        \item $\eta_A$ is the singleton map $a \mapsto  \{a\}$
        \item $\mu_A(X)$ is the big union $\bigcup X$
    \end{itemize}
\end{example}

\subsection{Lawvere Theories}
The semantics of algebraic effects have a categorical basis as
"countable enriched Lawvere theories freely generated by $\dots$ operations and equations".
\cite{plotkin2004computational}
"In mathematical practice Lawvere theories arise
whenever one has a functor into a category with finite products
and one studies the natural transformations between finite products of the functor".
\cite{hyland2007category}
A more in-depth mathematical background of Lawvere theories is,
regrettably, beyond the scope of this document.

Returning to the issue of effects,
the essential difference between monads and algebraic effects is that
where monads have a constructed object $TX$,
i.e. the computations of the type $X$,
algebraic effects have operations from which effects arise;
and thus $TX$ is derived not constructed.
Consider the monad $IO Int$ which is the object of computations
of the type $Int$;
from an algebraic effect perspective
one has the operation \texttt{print} which generates the effect.\\

\begin{example}
    An example of the Lawvere Theory $L_{I/O}$ for input/output
    is \textit{generated} by the operations
    \begin{equation}
        read : I \rightarrow 1 \quad\textrm{and}\quad write : 1 \rightarrow O
    \end{equation}
    where $I$ is a countable set of inputs and $O$ of outputs\cite{plotkin2001adequacy}.
\end{example}

We must also state that
a Lawvere theory is generated by operations subject to equations.

Whilst we have covered operations we have not covered
equations

whilst
operations appear directly in describing programming languages
equations do not.
One of the equations for side-effects is
\begin{equation}
    updateloc,\ v (updateloc,v\prime (x)) = updateloc,v\prime (x)
\end{equation}
and the corresponding program assertion is
\begin{equation}
    (l := x;\ let\ y\ be\ !l\ in\ M) = (l\ := x;\ M[x/y])
\end{equation}
This example is from \cite{plotkin2001adequacy}.

"a class of theories that can be viewed as categories with a monad,
so that
any category with a monad is, up to equivalence (of categories with a monad), one of such theories.
Such a reformulation in terms of theories is more suitable for formal manipulation and more appealing to those unfamiliar with Category Theory."
\cite{moggi1991notions}

The Lawvere theory LE for exceptions is the free Lawvere theory generated by E operations $raise : 0 \rightarrow  1$,
where E is a set of exceptions.
In terms of operations and equations,
this corresponds to an E-indexed family of nullary operations with no equations.
%The monad on Set generated by LE is TE = − + E. More generally, the forgetful functor UL : Mod(LE,C) C induces the monad
%−+E on C, where E is the E-fold copower of 1, i.e.,  exists in C.
It is shown in [39]
that this countable Lawvere theory induces Moggi’s side-effects monad $(S \times \textrm{-})^S$ on
Set.
\cite{hyland2007category}


\pagebreak
\section{Monads in the Wild}
Typically a programmer will think of a monad as three function signatures
which correspond to the triple $(T,\eta,\mu)$;
of course much more informally.
Firstly \texttt{fmap} corresponds to the functor $T$,
then we have \texttt{return} and \texttt{join} which correspond to
the natural transformations $\eta$ and $\mu$ respectively.
We should note that the terms \texttt{fmap,return,join}
are not fixed, relatively often other names are used;
for example \texttt{map} instead of \texttt{fmap}
or \texttt{pure/unit}instead of \texttt{return},
however the semantics are exactly the same.

\begin{equation}
  \begin{split}
    fmap   &:: M x \rightarrow (x \rightarrow y) \rightarrow M y \\
    return &:: x \rightarrow M x                                 \\
    join   &:: M (M x) \rightarrow M x
  \end{split}
\end{equation}

To be a faithful monad implementation it must be true that:
\begin{equation}
  \begin{split}
      \lambda f.\lambda x.return\ (fmap\ f\ x)
      &\equiv
      \lambda f.\lambda x.fmap\ (return \circ f)\ x)
      \\
      \lambda f.\lambda x.fmap\ f\ x
      &\equiv
      \lambda f.\lambda x.fmap\ f\ (fmap\ id\ x)
      \\
      \lambda f.\lambda x.join\ (join\ (fmap\ f\ x))
      &\equiv
      \lambda f.\lambda x.join(fmap\ f\ (join\ x))
  \end{split}
\end{equation}

Contrast these monad implementation laws with the diagrams
\begin{equation}
    \begin{tikzcd}[sep=large]
        T \rar{\eta_T} \drar[swap]{1_{C}} & T^2 \dar{\mu} & \lar[swap]{T\eta} \dlar{1_{C}} T \\
                                           & T            &
    \end{tikzcd}
    \quad
    \begin{tikzcd}[sep=large]
        T^3 \rar{T\mu} \dar[swap]{\mu_T} & T^2 \dar{\mu} \\
        T^2 \rar[swap]{\mu}                    & T
    \end{tikzcd}
\end{equation}

Do programming monads need to follow the same rules as category theory monads?
There is no way for the compiler to check these rules so practically the answer is no,
however it is wise for monad implementations to obey them so that
we can chain computations in predictable and understandable ways.

For programmers the essence of a monad is \texttt{fmap},
this is because it is our primary way to interact with the monad.
Typically we describe our desired result in terms of functions on the monad
done one after the other.
However one often sees \texttt{bind} where bind is of type
\begin{equation}
    bind :: M x \rightarrow (x \rightarrow M y) \rightarrow M y
\end{equation}
The reason why bind is as prominent,
if not more prominent, than \texttt{fmap}
is that it allows us to chain computations in a manner
which occurs more often that it does with \texttt{fmap}
%TODO
Bind is a hugely useful tool for programmers;
because often we want to sequence
Can we consider bind as a morphism between two objects in the Kleisli Category.
%TODO
\begin{equation}
    argument\
    \stackrel{f}{\rightarrow}     T\ y
    \stackrel{T g}{\rightarrow}   T\ T\ z
    \stackrel{\mu z}{\rightarrow} T\ result
\end{equation}

It is convention to use bind as the infix operator "$\bind$" where
\begin{equation}
    mx \bind f \equiv join\ (fmap\ f\ mx)
\end{equation}
Consider that \texttt{join} can be expressed in terms of \texttt{bind} (with \texttt{id})
\begin{equation}
    join\ m \equiv m \bind id
\end{equation}
and furthermore so can \texttt{fmap} (with \texttt{return}).
\begin{equation}
    fmap\ f\ x \equiv (return\ x) \bind f
\end{equation}

Because \texttt{bind} is often found with \texttt{(fmap,join,return)}
We can reformulate our axioms in terms of bind,
indeed the axioms are often reformulated this way because it is simpler.
\begin{equation}
  \begin{split}
        return\ x \bind f &\equiv f x \\
           m \bind return &\equiv m   \\
      (m \bind f) \bind g &\equiv m \bind (\lambda x.(fx \bind g))
  \end{split}
\end{equation}

We can further rephrase ourselves in terms
of the \textit{Kleisli composition operator} $\gg$ where
\begin{equation}
    \gg :: (x \rightarrow M y) \rightarrow (y \rightarrow M z) \rightarrow (x \rightarrow M z)
\end{equation}
and we have
\begin{equation}
    (f \gg g) \gg h \equiv f \gg (g \gg h)
\end{equation}
\begin{equation}
    return \gg f \equiv f \equiv f \gg return
\end{equation}

\subsection{State Monad Example}
Firstly recall our example \ref{lc1}
computations with side-effects:
\begin{itemize}
    \item $T(-)$ is the functor $(-\times S)^S$, where $S$ is a nonempty set of stores.
        Intuitively a computation takes a store and returns a value together with the modified store.
    \item $\eta_A$ is the map $a \rightarrow (\lambda s:S.\langle a,s \rangle)$
    \item $\mu_A$ is the map $f \rightarrow (\lambda s:S.eval(fs))$,
        i.e. $\mu_A(f)$ is the computation that given a store $s$,
        first computes the pair computation-store $\langle f\prime,s\prime\rangle = fs$
        and then returns the pair value-store $\langle a,s\prime\prime\rangle = f\prime s\prime$.
\end{itemize}

This translates to the code
this code is from the Haskell standard library

\begin{verbatim}
newtype State s a = State {runState :: s -> (a, s)}

join :: (State s (State s a)) -> (State s a)
join xss = State (\s -> uncurry runState (runState xss s))

instance Monad (State s) where
    return x = State (\ s -> (x,s))
    m >>= f  = State (\ s ->
        let (x,s') = runState m s in
        runState (f x) s')

class Monad m => MonadState s m | m -> s where
    -- | Return the state from the internals of the monad.
    get :: m s
    get = state (\s -> (s, s))

    -- | Replace the state inside the monad.
    put :: s -> m ()
    put s = state (\_ -> ((), s))

    -- | Embed a simple state action into the monad.
    state :: (s -> (a, s)) -> m a
    state f = do
      s <- get
      let ~(a, s') = f s
      put s'
      return a

type GameValue = Int
type GameState = (Bool, Int)

playGame :: String -> State GameState GameValue
playGame []     = do
    (_, score) <- get
    return score

playGame (x:xs) = do
    (on, score) <- get
    case x of
         'a' | on -> put (on, score + 1)
         'b' | on -> put (on, score - 1)
         'c'      -> put (not on, score)
         _        -> put (on, score)
    playGame xs

startState = (False, 0)

main = print $ evalState (playGame "abcaaacbbcabbab") startState
\end{verbatim}

\subsection{Monads for Structuring Programs}
In this section I will illustrate how effective monads are when used to structure programs.
\cite{jones1995functional}

Firstly one should note that OCaml has non-nullable types
i.e; one will never see null where one is expecting an
int or a binary tree or anything else.
Null values are always explicit.
The canonical representation of null is the None variant.
Why is this?
Why do we need monads to this?
What do other languages do?
Monads can be used to succinctly and expressively structure computation with the option type.

\begin{verbatim}
  type 'a option = Some 'a | None
  val return : 'a -> 'a option
  val join   : 'a option option -> 'a option
  val fmap   : 'a option -> ('a -> 'b) -> 'b option
\end{verbatim}

The code for these three functions is simple and fairly self-evident.
However from this simple basis we can construct much more complicated programs which we
will be certain will never have a \textit{NullPointerException}.
Furthermore the type system will ensure 
It will refuse to compile nonsensical code which does not type-check.
Such an assurance is invaluable in creating correct programs.
Here it's important to note that the effect is the \textit{NullPointerException};
we deal with it much more cleanly and effectively 100\% of the time using monads.

\begin{verbatim}
  let return a = Some a
  let join = function
      | Some (Some a) -> Some a
      | _ -> None
  let map a f = match a with
      | None -> None
      | Some b -> f b
\end{verbatim}

Consider this example code for searching a trie data structure.
Briefly, a trie is key value data structure;
where the key is a finite sequence of values
(typically a string which is equivalent to a sequence of characters).
Each node has an option value and a list of children.
The root of the trie represents the empty string.
Two auxiliary functions \texttt{find\_child} and \texttt{val\_extract}
are used in the search code; \texttt{val\_extract}
simply returns the first value in a pair,
find child searchs the list of children returning
the node with the matching character given;
that is only should that child exists.

\begin{verbatim}
  type ('k, 'v) t = Trie of 'v option * (('k * ('k, 'v) t) list)
  val find_child  : ('k, 'v) t -> 'k -> ('k, 'v) t option
  val val_extract : ('k, 'v) t -> 'v option
  val create      : 'k list -> 'v -> ('k, 'v) t
  val get         : ('k, 'v) t -> 'k list -> 'v option

  let create key data =
    let rec aux = function
      | []      -> Trie (Some data, [])
      | c :: cs -> Trie (None, [(c, aux cs)])
    in aux key
\end{verbatim}

Create is shown to just illustrate the data structure,
we simply iterate across the list creating a node in the trie for each character
until the list is exhausted at which point we insert the value.

%TODO tikz example of the data structure

In the search code t is the trie, key is a list of characters over which
we iterate the search function. For each character we call find child on
the current node. Find child, if succesful, will return the next node
upon which we continue the search with the remaining characters. Once we
are at the last character we know to try and extract the value from the end
node. The key point here is that using bind allows us to succintly only code
for the happy case but deal with the error case at every step.

\begin{verbatim}
  let get t key =
    let rec search chars t =
        match chars with
        | []      -> val_extract t
        | c :: cs -> bind_search (find_child t c) cs
    and bind_search ot chars = ot >>= search chars
    in search key t
\end{verbatim}

The result type is similar to the option type, however we use an extra type
parameter for the unhappy case, essentially the result type encapsulates 
either an error or a correct computation result. The result monad
corresponds to a generalisation of the exception monad presented by Wadler \cite{wadler1995monads}.

\begin{verbatim}
  type ('a, 'e) result = Ok of 'a | Error of 'e
  val bind : ('a, 'e) result
          -> ('a -> ('b, 'e) result)
          -> ('b, 'e) result
\end{verbatim}

In this example we have a list assocation which is a list of pairs, in this case both
items in the pairs are strings.

\begin{verbatim}
  let list_assoc_to_job la =
    let find k = match List.Assoc.find la k with
      | None -> Error (Err.missing_key k)
      | Some v -> Ok v
    in
    find "name"   >>=                  (fun name   ->
    find "prog"   >>=                  (fun prog   ->
    find "args"   >>= parse_args   >>= (fun args   ->
    find "run_at" >>= parse_run_at >>= (fun run_at ->
      Ok (Job.create name prog args run_at ())
    ))))
\end{verbatim}

Bind is used to succintly short circuit a computation when a value can not be
correctly obtained. As such this allows the program to be structured neatly to return
an error with precise information for which key could not be found or which value could
not be parsed. The result monad is very similar to the option monad. It would be interesting
to examine whether algebraic effects can be used to structure a program in a similar manner.

Typically programming languages allow the elison of the anonymous function on the right hand side
of the bind to simply introduce the binding into the environment;
in this case OCaml ppx extension points are used in the form \textit{let\%bind}.
This is largely just a programmer convienence,
for comparison Haskell offers \textit{do notation} for the same end.

\begin{verbatim}
  let list_assoc_to_job la =
    let find k = match List.Assoc.find la k with
      | None -> Error (Err.missing_key k)
      | Some v -> Ok v
    in
    let%bind name   =  find "name"                     in
    let%bind prog   =  find "prog"                     in
    let%bind args   = (find "args"   >>= parse_args  ) in
    let%bind run_at = (find "run_at" >>= parse_run_at) in
    Ok (Job.create name prog args run_at ())
\end{verbatim}
Is equivalent to
\begin{verbatim}
  let list_assoc_to_job la =
    let find k = match List.Assoc.find la k with
      | None -> Error (Err.missing_key k)
      | Some v -> Ok v
    in
    find "name"   >>=                  (fun name   ->
    find "prog"   >>=                  (fun prog   ->
    find "args"   >>= parse_args   >>= (fun args   ->
    find "run_at" >>= parse_run_at >>= (fun run_at ->
      Ok (Job.create name prog args run_at ())
    ))))
\end{verbatim}

\texttt{do} is simply syntactic sugar for \texttt{>>=} and \texttt{>>} where
\begin{equation}
    \texttt{>> :: M x -> M y -> M y}
\end{equation}
i.e. do first and ignore the result and then do the second.
\begin{align}
    \texttt{do}\ \{ x;\ \texttt{<stmts>} \}
    &\equiv x \texttt{ >> do \{<stmts>\}}
    \\
    \texttt{do}\ \{ v \leftarrow x\ \texttt{ <stmts>}\}
    &\equiv x \texttt{ >>= } \lambda v.\ \texttt{do}\ \{ \texttt{<stmts>} \}
    \\
    \texttt{do}\ \{\texttt{let }x = v\ \texttt{<stmts>}\}
    &\equiv (\lambda x.\ \texttt{do}\ \{ \texttt{<stmts>} \})v
\end{align}

In this example we simply and effectively deal with two types of errors.
Firstly a key being missing from our list
\begin{verbatim}
    let find k = match List.Assoc.find la k with
      | None -> Error (Err.missing_key k)
      | Some v -> Ok v
\end{verbatim}
Secondly a more complicated value which we have to parse;
the parsing of which can fail
\begin{verbatim}
    let%bind run_at = (find "run_at" >>= parse_run_at) in
\end{verbatim}
The key point here is when we have an error we want to fail
loudly and give as much information as possible about why we failed.
Furthermore, we don't want to take care of that in this function
we don't want to put that logic here
Monads allow us to sequence this error handling
and keep the necessary logic in a sensible re-usable place.

Finally consider the relation between
\texttt{(let\%bind,do)} and the Kleisli category of programs.
Consider how convienent it is for programmers to have a category
of programs where reliable error handling is built in by default.

%%TODO
\subsection{Monads for Imperative Programming}
Imperative programming is undeniably popular,
and much closer to how computers physically work than functional programming;
There is a bridge to be crossed between functional and imperative programming

monads have been concretely shown to be an effective method of replicating
imperative style and retaining benefits of purity \cite{PeytonJones:1993}.
These examples clearly illustrate how useful monads can be for structuring a program,
however the importance and necessity of monads is that for pure functional languages
to be actually useful we require effects. Monads give us means to achieve that.

In this section we will prove the inverse,
monads are good for imperative programming
and imperative programmers have monad-like features



"Moreover, effect handlers allow con- current programs to be written in direct-style retaining the simplicity of sequential code as opposed to callback-oriented style with either monadic concurrency libraries such as Lwt [8] and Async [5] for OCaml or explicit callbacks."
\cite{dolaneffectively}

"The use of monads to structure functional programs is de- scribed. Monads provide a convenient framework for simulating effects found in other languages, such as global state, exception handling, out- put, or non-determinism. Three case studies are looked at in detail: how monads ease the modification of a simple evaluator; how monads act as the basis of a datatype of arrays subject to in-place update; and how monads can be used to build parsers.
"
"Pure functional languages have this advantage: all flow of data is made explicit. And this disadvantage: sometimes it is painfully explicit."
"It is with regard to modularity that explicit data flow becomes both a blessing and a curse. On the one hand, it is the ultimate in modularity. All data in and all data out are rendered manifest and accessible, providing a maximum of flexibility. On the other hand, it is the nadir of modularity. The essence of an algorithm can become buried under the plumbing required to carry data from its point of creation to its point of use"
\cite{wadler1995monads}

"
%Do es the monadic style force one􏰄 in e􏰒ect􏰄 to write a functional facsimile of an
%imp erative program, thereby losing any advantages of writing in a functional language?
%the p ower of higher􏰆order functions and non􏰆strict semantics can b e used
%programming easier􏰄 by de􏰐ning new action􏰆manipulating combinators􏰇
%The p oint we are making is that it is easy for the programmer to de􏰐ne new 􏰜glue􏰝 to combine actions in just the way which is suitable for the program b eing written􏰇 It􏰚s a bit like b eing able to de􏰐ne your own control structures in an imp erative language􏰇
"\cite{PeytonJones:1993}


"The special characteristics and advantages of functional programming are often summed up more or less as follows. Functional programs contain no assignment statements, so variables, once given a value, never change. More generally, functional programs contain no side-effects at all. A function call can have no effect other than to compute its result. This eliminates a major source of bugs, and also makes the order of execution irrelevant - since no side-effect can change the value of an expression, it can be evaluated at any time. This relieves the programmer of the burden of prescribing the flow of control."
"When writing a modular program to solve a problem, one first divides the problem into sub- problems, then solves the sub-problems and combines the solutions. The ways in which one can divide up the original problem depend directly on the ways in which one can glue solutions together. Therefore, to increase ones ability to modularise a problem conceptually, one must provide new kinds of glue in the programming language."
\cite{hughes1989functional}

In the Go programming language one will often see this idiom,
one can easily see the relation to the result monad which we
have demonstrated.
We get this for free with monads!
Go does not have exceptions,

\begin{verbatim}
    result, err := myFunction()
    if err != nil {
        return nil, err
    }
\end{verbatim}

Consider our error handling example;
in Go this might be much more verbose
and perhaps painfully explicit
Of course one would probably structure the program differently in Go;
however the OCaml implementation is undeniably straightforward
and error resistant, while providing useful error messages.

\begin{verbatim}
    name, err := findName(la)
    if err != nil {
        return nil, err
    }
    prog, err := findProg(la)
    if err != nil {
        return nil, err
    }
    etc...
    job.Create(name, prog, args, run_at)
\end{verbatim}

\subsection{Monad Transformers}
The elephant in the room

Now we must discuss how one composes monads;
say

Monad transformers add functionality of one monad to another.
We often find that difficulty arises when trying to compose monads,
say we have a stateful computation a

"Monad transformers add functionality of one monad to another. They do so by stacking a transformer version of the additional monad on top of the original one, which with enough repetition results in something like a monadic Voltron."
"Formally, monad transformers are illustrated well by the lift function: the only function in the MonadTrans class. For regular monads its type signature looks like this:

liftM :: Monad m => (a -> r) -> m a -> m r
whereas for monad transformers its type signature looks like this:

lift :: Monad m => m a -> t m a
where m is any monad and t is the transformer."



"Monad transformers have an inherent limitation: they enforce the static ordering of effect layers and hence statically fixed effect interactions. There are practically significant computations that require interleaving of effects. ‘Delimited Dynamic Binding’ (ICFP 2006) was first to bring up this point. The ‘Extensible Effects’ paper expanded that discussion on new examples. Section 5 describes simple and common programming patterns that are particularly problematic with monad transformers because the static ordering of effect layers is not flexible.

"

"A monad transformer is similar to a regular monad, but it's not a standalone entity: instead, it modifies the behaviour of an underlying monad. Most of the monads in the mtl library have transformer equivalents. By convention, the transformer version of a monad has the same name, with a T stuck on the end. For example, the transformer equivalent of State is StateT; it adds mutable state to an underlying monad. The WriterT monad transformer makes it possible to write data when stacked on top of another monad."

"As we have already mentioned, when we stack a monad transformer on a normal monad, the result is another monad. This suggests the possibility that we can again stack a monad transformer on top of our combined monad, to give a new monad, and in fact this is a common thing to do. Under what circumstances might we want to create such a stack? 1 comment

If we need to talk to the outside world, we'll have IO at the base of the stack. Otherwise, we will have some normal monad. 4 comments

If we add a ReaderT layer, we give ourselves access to read-only configuration information. No comments

Add a StateT layer, and we gain global state that we can modify. 2 comments

Should we need the ability to log events, we can add a WriterT layer. No comments

The power of this approach is that we can customise the stack to our exact needs, specifying which kinds of effects we want to support. No comments

As a small example of stacked monad transformers in action, here is a reworking of the countEntries function we developed earlier. We will modify it to recurse no deeper into a directory tree than a given amount, and to record the maximum depth it reaches.

In the framework that mtl provides, each monad transformer in the stack makes the API of a lower level available by providing instances of a host of typeclasses. We could follow this pattern, and write a number of boilerplate instances.


In this case, the only way we can access the underlying State monad's put is through use of lift. No comments

\begin{verbatim}
-- file: ch18/StackStack.hs
innerPut :: String -> Foo ()
innerPut = lift . put
\end{verbatim}

Stacking monad transformers is analogous to composing functions. If we change the order in which we apply functions, and we then get different results, we are not surprised. So it is with monad transformers, too.

Finally, when we use monads and monad transformers, we can pay an efficiency tax. For instance, the State monad carries its state around in a closure. Closures might be cheap in a Haskell implementation, but they're not free.
"
\cite{o2008real}









Note that OCaml is impure and doesn't have monad transformers,
it doesn't need them.
Why?
How does this interact with modules composing?



\cite{king1993combining}
%describe how some monads may b e combined with others to yield a combined monad􏰀
%For our purp oses􏰂 we will think of a monad as a typ e constructor􏰂 together with three functions that must satisfy certain laws􏰀 For instance􏰂 we may have an interpreter and wish it to return􏰂 not just a value􏰂 but the numb er of reduc􏰉 tion steps taken to reach the value􏰀 Later􏰂 we may want our interpreter either to return a value or an error message􏰀 Without using a monad or a similar discipline􏰂 it would be a messy undertaking to do in a purely functional pro􏰉 gramming language􏰀 If however our interpreter was written in a monadic style􏰂 we would merely have to change the monad to change what extra information the interpreter should return􏰀


%are now two distinct approaches to combining E x with S t􏰂 either􏰊
%type S tE x a 􏰏 S tate 􏰝 􏰌E x a􏰚 S tate􏰎
%type E xS t a 􏰏 S tate 􏰝 E x 􏰌a􏰚 S tate􏰎
%these have entirely di􏰓erent meanings􏰀 In the 􏰁rst mo del􏰂 an expression that raises an exception also returns a state􏰀 This monad may b e thought of as mo delling the language Standard ML􏰂 where when an exception is raised the state survives􏰀 In the second mo del􏰂 if an expression raises an exception then
%the
%state is not returned􏰀



\pagebreak
\section{Algebraic Effects}
Following from our discussion of Lawvere theories,
we will now explore their programming language progeny \textit{algebraic effects}.
The essence is we have \textit{operations} which generate effects,
and to actually interact with them as programming constructs
we need \textit{handlers}\cite{plotkin2009handlers};
which, informally, deconstruct effects.
This section will cover these two issues and their use.

We can begin to understand algebraic effects by saying that
we have effects we can manipulate algebraically;
that is the interpretation of an expression is
subject to an operator and its subexpressions.
For instance, $a + b$ is interpreted as an operator
$+$ with its own semantics along with the value of its
two subexpressions $a$ and $b$.
One can consider \texttt{panic}, as a nullary operator
with no subexpressions.

We can further our initial understanding with another view
relating algebraic effects to continuations,
or rather algebraic effects being given by continuations.
Whenever an operation is performed we have a continuation
passed to the handler;
which the handler uses to "implement" that effect.
This view also allows us an initial but intuitive
understanding of why algebraic effects are incompatible with continuations;
it is not possible to distribute global control flow across several mechanisms.

However algebraic effects are not simply just continuations,
they are programmer defined effects.
To the programmer perhaps this is perhaps the most interesting implication,
we can now have programmer defined effects,
moreover potentially specifying multiple handlers for each effect.
Previously effects were completely implicit,
and this is a large reason why they can cause bugs,
now they are under direct and explicit control of the programmer.

Currently algebraic effects are a popular area of research,
as such we have seen a multitude of implementations in recent years.
They may well enter the mainstream (as mainstream as monads at least)
due to multicore OCaml being implemented in terms of algebraic effects.
Additionally, it is likely that imperative programmers
will find algebraic effects more intuitive than an equivalent monads.

\textbf{Some Algebraic Effect Implementations}
\begin{itemize}
    \item Eff: Matija Pretnar and Andrej Bauer,
        2015\cite{bauer2015programming}
    \item Frank: Sam Lindley and Conor McBride
        2016\cite{Lindley:2016vz}
    \item OCaml+effects: Stephen Dolan, Leo White, KC Sivaramakrishnan,
        2016\cite{ocamlplseff}
    \item Koka: Dan Leijen
        2016\cite{leijen:16}
\end{itemize}

\subsection{Overview}
\begin{example}
    Recall the examples of effects we gave at the beginning of this document,
    each of these effects has a natural operation which generates the effect.
    Consider
    \begin{itemize}
        \item Input/Output \textit{generated by} \texttt{read, write}
        \item Mutable state \textit{generated by} \texttt{set, get}
        \item Exceptions \textit{generated by} \texttt{raise}
        \item Non-Determinism \textit{generated by} \texttt{choose}
        \item Probabilistic Non-Determinism \textit{generated by} \texttt{choice}
    \end{itemize}
\end{example}

Notice that continuations are not mentioned here,
this is because, as discussed previously,
continuations are of a computationally different character
than these other effects
\cite{Plotkin:2002dw}\cite{hyland2007combining}.

Let us examine input and output as an algebraic effect first,
we can begin with the quintessential programming example;
"Hello world!".
We will be working in OCaml+effects for this example.\\

\begin{example}
    Consider this example of an IO effect,
    first we define an effect \texttt{IO},
    this is, in OCaml, similar to defining
    a new exception type.
    We then describe two constructors
    \texttt{Print} and \texttt{Read}
    of the effect \texttt{IO}.

    \begin{verbatim}
        effect IO =
          | Print : string -> unit
          | Read : string
    \end{verbatim}

    Now to actually have an effect we need to use the primitive \texttt{perform},
    we construct an effect via \texttt{(Print "Hello world!")}.

    \begin{verbatim}
        let print_hello =
            perform (Print "Hello world!")
    \end{verbatim}

    Because OCaml is impure, the handler of this effect looks rather obtuse,
    however in a pure or lazy language this effect and handler would be necessary.
    Here it is just an indirection for the sake of an example.

    \begin{verbatim}
        let () =
          try print_hello with
          | effect (Print s) k -> printf s; continue k
    \end{verbatim}
\end{example}

%TODO
Another example of algebraic effects is binary non-determinism,
here we simulate a coin flip.
\begin{example}
    \begin{verbatim}
    effect choice =
        | Choose : bool

    let coin_flip () =
        if (perform Choose) then
            Heads
        else
            Tails

    try coin_flip () with
    | effect Choose k -> continue k (Random.bool ())
    \end{verbatim}
\end{example}

The core benefits of algebraic effects are
firstly that composition is simple and pain free;
we will examine this later.

Secondly they allow the programmer to
"separate the expression of an effectful computation from its implementation."
They "provide a modular abstraction for expressing effectful computation"\cite{dolan2015effective},
\cite{dolan2015effective}

We saw a similar benefit with monads,
in particular we could seperate the happy path from our error handling.
It would appear that this seperation is even more general with AE

It would appear that algebraic effects are particularly popular
for systems programming
\cite{dolan2015effective}
\cite{dolan2017concurrent}
\cite{dolaneffectively}
\cite{Dolan:2017}
we will try to elucidate why this,
especially in comparison to monads.
"In this paper, we make the observation that effect handlers
can elegantly express particularly difficult programs
that combine system programming and concurrency without compromising performance."
\cite{Dolan:2017}

\subsection{Handlers}
Thus far we have only
where operations are constructors of effects
we also need a dual,
\textit{handlers} which can be seen as deconstructors of effects
\cite{}.
Handlers were introduced by Plotkin and Pretner\cite{Plotkin:2009dr}
builds on
\cite{benton2001exceptional}.

new handling construct generalises the exception-handling construct of Benton and Kennedy. \cite{benton2001exceptional}

"Of the various operations, handle is of a different computational character and,
although natural, it is not algebraic
Andrzej Filinski (personal communication) describes handle as a deconstructor,
whereas the other operations are constructors (of effects)"
\cite{Plotkin:2002dw}

"present an algebraic treatment of exception handlers and,
more generally, introduce handlers for other computational effects
representable by an algebraic theory"
"Although the algebraic approach has given ways of constructing,
combining, and reasoning about effects, it has not yet accounted for their handling.


The difficulty is that exception handlers, a well-known programming concept,
fail to be algebraic operations.

Conceptually, algebraic operations and effect handlers are dual:
the former could be called effect constructors as they give rise to the effects;
the latter could be called effect deconstructors as they depend on the effects already created.

Filinski’s reflection and reification operations provide general effect constructors
and deconstructors in the context of layered monads [5]."

"The central new semantic idea is that deconstructing a computation amounts to
applying to it a unique homomorphism guaranteed by universality.
The domain of this homomorphism is a free model of the algebraic theory of the effects at hand;
its range is a programmer-defined model of the algebraic theory;
and it extends a programmer-defined map on values."
\cite{Plotkin:2009dr}


"Plotkin and Pretnar’s handlers for algebraic effects occupy a sweet spot in the design space of abstractions for effectful computation. By separating effect signatures from their implementation, alge- braic effects provide a high degree of modularity, allowing pro- grammers to express effectful programs independently of the con- crete interpretation of their effects. A handler is an interpretation of the effects of an algebraic computation. The handler abstraction adapts well to multiple settings: pure or impure, strict or lazy, static types or dynamic types."
\cite{kammar2013handlers}

A key benefit of handlers, and hence of algebraic effects,
is that

Generalised exception handlers. Benton and Kennedy [3] intro- duced the idea of adding a return continuation to exception han- dlers. Their return continuation corresponds exactly to the return clause of an effect handler. Effect handler operation clauses gener- alise exception handler clauses by adding a continuation argument, providing support for arbitrary effects. An operation clause that ig- nores its continuation argument behaves like a standard exception handler clause.

and there is a natural separation between
their interface (as a set of operations)
and
their semantics (as a handler)."
"the signature of the effect operations forms a free algebra which gives rise to a free monad.
Free monads provide a natural way to give semantics to effects,
where handlers describe a fold over the algebra of operations.
Using a more operational perspective,
we can also view algebraic effects as resumable exceptions
(or perhaps as a more structured form of delimited continuations)."
\cite{leijen2017type}

\subsection{Composing Algebraic Effects}
What has become the most evident deficiency of monads is
how difficult and tricky it is to compose them.
This is particularly unpleasant as
composability is often touted as one the most practical advantages of functional programming.
Fortunately composing algebraic effects is much less complicated than composing monads!
There exists an intuitive way to compose algebraic effects,
for example one doesn't need to consider order of composition as one does with monads.
This advantage stems from the theory in fact,
the "notion of countable enriched Lawvere theory
provides us with a natural way to describe
how computational effects may be combined
"\cite{plotkin2004computational}.

Now let us tackle the question of how to actually compose
algebraic effects.\\

\begin{example}
    Recall our IO and non-determinism examples,
    we can combine them with ease!
    \begin{verbatim}
        effect IO =
          | Print : string -> unit
          | Read : string

        effect choice =
          | Choose : bool

        effect int_state =
          | Get : int
          | Set : int -> unit

        let incr_maybe_twice () : int =
          perform (Set ((perform Get) + 1));
          if (perform Choose)
          then perform (Set ((perform Get) + 1))
          else perform (Print "incremented just once");
          perform Get
    \end{verbatim}
    We simply need to add a handler for each effect we incorporate.
    \begin{verbatim}
          match incr_maybe_twice () with
          | result -> result
          | effect (Put s') k -> (fun s -> continue k () s')
          | effect Get k -> (fun s -> continue k s s)
          | effect (Print s) k -> printf s; continue k
          | effect Choose k -> continue k (Random.bool ())
    \end{verbatim}
\end{example}

\subsection{Structuring Programs with Algebraic Effects}
As we demonstrated monads are rather useful for structuring programs,
for example,
allowing us to only consider the happy path while remaining assured
edge cases are handled;
it is natural to ask if programming with algebraic effects
has the same properties?

Recall our implementation of a trie with monads,
we will now explore whether an algebraic effect implementation
\begin{verbatim}
  let get t key =
    let rec search chars t =
        match chars with
        | []      -> val_extract t
        | c :: cs -> (find_child t c) >>= searh cs
    in search key t
\end{verbatim}

Contrast this with an algebraic effect implementation

\begin{verbatim}
  effect exception = Exn

  let get t key =
    let rec search chars t =
        match chars with
        | []      -> val_extract t
        | c :: cs -> search_children t c cs
    and search_children t c chars =
        match find_child t c with
        | None -> perform Exn
        | Some -> search chars
    in try search key t with
    | effect Exn _ -> None
    | value -> value
\end{verbatim}

We can immediately see that this implementation is more verbose,
which is not necessarily desirable;
however the mechanisms of how this function works
are more apparent.

Next we shall consider an example
To illustrate the purpose of this function here
is an example of two variables.
It generates $2^n$ where $n = 2$

\begin{verbatim}
    type t =
        | And   of t * t
        | Or    of t * t
        | Not   of t
        | Var   of var
        | Const of bool
    type var = string
    type ctxt = (var * bool) list

    gen_ctxts ["a";"b"];;
    - : (string * bool) list list = [
        [("a", true);  ("b", true)];
        [("a", true);  ("b", false)];
        [("a", false); ("b", true)];
        [("a", false); ("b", false)]
    ]
\end{verbatim}

Here is a straightforward implementation
\begin{verbatim}
  let gen_ctxts vars =
    let push ctxt v b = List.map ctxt ~f:(fun x -> (v, b) :: x) in
    let rec aux vs = match vs with
      | [] -> []
      | [v] -> [[v, true]; [v, false]]
      | v :: vs ->
        let vs' = aux vs in
        (push vs' v true) @ (push vs' v false)
    in aux vars
\end{verbatim}

\begin{verbatim}
  let gen_ctxts vars : var list -> (var * bool) list list =
    let push ctxt v b = List.map ctxt ~f:(fun x -> (v, b) :: x) in
    let rec aux vs = match vs with
      | [] -> []
      | [v] -> [[v, true]; [v, false]]
      | v :: vs ->
        let vs' = aux vs in
        (push vs' v true) @ (push vs' v false)
    in aux vars
\end{verbatim}
Consider the algebraic effect version
%TODO
%TODO
%TODO
%TODO
%TODO

\begin{verbatim}
  let gen_ctxts vars =
    let push ctxt v b = List.map ctxt ~f:(fun x -> (v, b) :: x) in
    let rec aux vs = match vs with
      | [] -> []
      | [v] -> [[v, true]; [v, false]]
      | v :: vs ->
        let vs' = aux vs in
        let b = perform Choose in
        push vs' v b
    in aux vars

    let all_results vars =
      match gen_ctxts vars with
      | result -> result
      | effect Choose k ->
         (continue k true) @ (continue (Obj.clone_continuation k) false)
         (* OCaml effects/multicore only supports single-shot
           continuations. But, we can simulate multi-shot continuations by
           copying a continuation (using Obj.clone) before invocation. *)
\end{verbatim}

\subsection{Comparison with Monads}
"
A fundamental problem with monad transformer stacks is that
once a particular abstract effect is instantiated,
the order of effects in the stack becomes concrete,
and it becomes necessary to explicitly lift operations
through the stack."
\cite{kammar2013handlers}
Monad transformers can quickly become unwieldy when there are lots of effects to manage,
leading to a temptation in larger programs to combine everything into one coarse-grained state and exception monad.
\cite{brady2013programming}
lots of ppl say the monads when composed make effects coarse grained
and this is ugly
OCaml is impure but has monads, and soon effects, wen you need them?
is this the most 'practical' approach?
algebraic effects are more scalable solution for engineering programs!
this maybe one of the reasons why they seem to be so popular in systems
programming


"Moreover, effect handlers allow concurrent programs to be written in direct-style retaining the simplicity of sequential code as opposed to callback-oriented style with either monadic concurrency libraries such as Lwt [8] and Async [5] for OCaml or explicit callbacks."
\cite{dolaneffectively}


\pagebreak
\section{Abridged version}
This document will introduce two concepts,
bindables and resumable exceptions,
and will try to show that they can do surprising things.

\subsection{Bindables}
Roughly, a bindable anything which supports an
interface of two functions $bind$ and $new$.
\begin{align}
    bind &:: B\ x \rightarrow (x \rightarrow List\ y) \rightarrow B\ y\\
    new &:: x \rightarrow B\ x
\end{align}
The way to read is that $::$ means "is of type",
the bind function takes two things,
firstly $B\ x$, which means any bindable of underlying type $x$
where $x$ can be any type you want.
Then it takes $(x \rightarrow B\ y)$ which
is a function taking type $x$,
to the type $B\ y$ which a bindable of type $y$
where $y$ is any type
(and potentially but not necessarily type $x$).
Having been given those two things,
$bind$ returns a bindable of type $y$.
You could think of $bind$ as a sequencer between bindables.

The $new$ function takes any value and
constructs a bindable on top of that value.
Bindables can be built on top of any value,
even a bindable of the same type
or a completely different bindable entirely
(which is of course built on some other value).

Let's give a concrete example,
the easiest and most ubiquitous example of a bindable are lists.
\begin{align}
    bind &:: List\ x \rightarrow (x \rightarrow List\ y) \rightarrow List\ y\\
    new &:: x \rightarrow List\ x
\end{align}
Now implementing $new$ is pretty easy.
\begin{verbatim}
    //Note we use 'unit' as return is a keyword
    const new = x => [x];
    console.log(new(1));
    // === Array [1]
\end{verbatim}

To define bind we gotta define an auxiliary function called join first,
it takes a list of lists and returns a list.
It basically just flattens the list given,
appending each list to the one before it.

\begin{verbatim}
    const arr2 = [[1, 2], [3, 4]];
    const append = (onto, val) => onto.concat(val);
    const join = x => x.reduce(append, []);
    console.log(join(arr2));
    // === Array [1, 2, 3, 4]
\end{verbatim}
Once we have done that, defining bind is pretty simple
\begin{verbatim}
    const bind = (x, f) => join(x.map(f));
    const sqrButKeep = x => [x, x*x];
    console.log(bind([2, -9, 49], sqrButKeep));
    // === Array [2, 4, -9, 81, 49, 2401]
\end{verbatim}

So these examples illustrate what the functions do,
but they are not terribly useful.
From these simple pieces we can create much
more complicated things.
For example, using bindables we can get get rid of null-pointer exceptions.
And exceptions in general!
They cleverly short circuit computations like \texttt{\&\&}.
They can be used to create state out of immutability.
They can even replace the most common Go idiom:
\begin{verbatim}
    result, err := myFunction()
    if err != nil {
        return nil, err
    }
\end{verbatim}

Let's explore removing exceptions.
Since Go already does not have exceptions
we can start with that and show how to improve it even further.
A result is either an error (of type $y$),
or a successful computation (of type $x$).
Having defined that we can create result bindable.
Here is what bind for result might look like in JS.
\begin{verbatim}
    let bind = (result, fn) => {
        if (result.isErr()) {
            return result;
        } else {
            return fn(result);
        }
    }
\end{verbatim}
Now given a sequence of Go-like computations,
any of which may fail,
\begin{verbatim}
    name, err := dbCall(init)
    if err != nil {
        return nil, err
    }
    ipAddr, err := tableLookUp(name)
    if err != nil {
        return nil, err
    }
    geoLoc, err := getLoc(name, ipAddr)
    if err != nil {
        return nil, err
    }
    return create_profile(name, geoLoc)
\end{verbatim}
We can rephrase our code using bind,
assuming that bind is a method of result type objects.
First we call $new$ on \texttt{init} so we can begin our
sequence.
\begin{verbatim}
    new(init).bind(fun init ->
        dbCall(init).bind(fun name ->
            tableLookUp(name).bind(fun ipAddr ->
                getLoc(name, ipAddr).bind(fun geoLoc ->
                    return create_profile(name, geoLoc)
                )
            )
        )
    )
\end{verbatim}
As it turns out we ended up with a pyramid of doom and not an improvement.
But using code (AST) processing tools we can get something much simpler,
\texttt{letbind} simply rewrites the code below into the code above.
\begin{verbatim}
    letbind name := dbCall(init)
    letbind ipAd := tableLookUp(name)
    letbind geoL := getLoc(name, ipAddr)
    return create_profile(name, geoLoc)
\end{verbatim}
This code is much simpler and allows us just as much
safety and expressiveness as the Go code.

\subsection{Resumable Exceptions}
A resumable exception is an extension normal exceptions
and exceptions handlers, except we get a continuation to play with.
We can implement async/await using resumable exceptions.
Or concurrency. Or as we mentioned, nearly
anything bindables can do.
Proving that resumable exceptions can replace normal
exceptions seems rather tautological,
as such we will explore stateful computation.

We firstly need a wrapper function which will
take an initial state and will play through
the state as it changes. At no point
do we mutate the state, we just pass an updated
state whenever it is requested.
\begin{verbatim}
function runState(statefulFn, initState) {
    let runState = () => try {
        statefulFn();
    } with {
      { type: "Value", value } => (state => (state, value)),
      { type: "Get", cont } => (curState =>
          continue(cont, curState)(curState))
      { type: "Put",   value: newState, cont } => (
          state => continue(cont)(newState))
    }
    runState()(initState);
}
\end{verbatim}
What happens here is that every time we catch something we build up a function,
for example \texttt{(state => (state, value))} means I will return
you a function which when given a state will give you back the value
and the state.
Similarly \texttt{(state => continue(cont)(newState))}
says I will continue where I was before and whatever that returns
is applied to the newState I was just told to use.
Eventually the initial state is passed in
and the entire function we built up evaluates.
This makes it seem like we have mutable state
where actually we just have different values
being passed between functions.

The \texttt{perform} function throws an effect/exception
which we catch and deal with according to it's type.
Each time it also gives our handler as continuation
so we can get back where we were previously.

Finally we can play through the computation
using our wrapper function,
some computation to run,
and an initial state.
\begin{verbatim}
let incrTwice = () => {
  perform("Set", (perform("Get") + 1));
  perform("Set", (perform("Get") + 1));
  return perform("Get");
}

runState(incrTwice, 0) // === 2
\end{verbatim}


\pagebreak
\section{Conclusion}
it is fundementally unarguably more difficult to reason about code with side effects
monads and algebraic effects help make this easier

\pagebreak
\appendix
\section{\\A Monadic Trie}
\begin{verbatim}
open Core

module Trie : sig
    type ('e, 'v) t

    val create   : key :'a list -> data : 'b -> ('a, 'b) t
    val add      : ('a, 'b) t -> key :'a list -> data : 'b -> ('a, 'b) t
    val get      : ('a, 'b) t -> key :'a list -> 'b option
    val is_entry : ('a, 'b) t -> key :'a list -> bool

end = struct
    type ('e, 'v) t = Trie of 'v option * (('e * ('e, 'v) t) list)

    let create ~key ~data =
        let rec aux = function
            | []      -> Trie (Some data, [])
            | c :: cs -> Trie (None, [(c, aux cs)])
        in aux key

    let val_extract      (Trie (v, _)) = v
    let children_extract (Trie (_, c)) = c

    let key_extract   (k, _) = k
    let child_extract (_, c) = c

    let find_child trie c =
        let cl = children_extract trie in
        let rec aux = function
            | [] -> None
            | t :: ts ->
                if c = (key_extract t)
                then Some (child_extract t)
                else aux ts
        in aux cl

    let get t ~key =
        let open Option.Monad_infix in
        let rec search chars t =
            match chars with
                | []      -> None
                | [c]     -> find_child t c >>= val_extract
                | c :: cs -> bind_search (find_child t c) cs
        (* Because find child is opt we want to short circuit sometimes *)
        and bind_search ot chars = ot >>= search chars
        in search key t

    let is_entry t ~key = get t ~key |> Option.is_some

    (* This function is find child but also returns other children seperately. *)
    let seperate_child children key =
        let rec aux fwd bwd = match fwd with
            | [] -> None
            | n :: ns ->
                if (key_extract n) = key
                then Some ((child_extract n), ns @ bwd)
                else aux ns (n :: bwd)
        in aux children []

    let add t ~key ~data =
        let rec aux t chars =
            match chars with
                | [] -> (* Set the data *)
                    let Trie (_, children) = t in
                    Trie (Some data, children)
                | c :: cs -> (* Descend trie via current char c *)
                    let Trie (v, children) = t in
                    Trie (v, (descend children c cs))
        and descend children ch cs =
            match seperate_child children ch with
            (* Create new child *)
            | None                 -> (ch, (create ~key:cs ~data)) :: children
            (* Descend child *)
            | Some (child, others) -> (ch, (aux child cs))         :: others
        in aux t key
end
\end{verbatim}

% the \\ insures the section title is centered below the phrase: AppendixA

Text of Appendix A is Here

\section{\\An Algebraic Effect Trie}

\pagebreak
\bibliographystyle{unsrt}
\bibliography{dissert}

\end{document}
