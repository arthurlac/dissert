\documentclass[a4paper,10pt]{article}
\usepackage{listings}
\usepackage[utf8]{inputenc}
\usepackage[english]{babel}
\usepackage{amsfonts}
\usepackage{amsmath}
\usepackage{amssymb}
\usepackage{amsthm}
\usepackage{mathtools}

\title{A comparison of algebraic effects and monads}
\author{Arthur}
\date{\today}

\setlength{\parindent}{1em}
\setlength{\parskip}{1em}

\theoremstyle{definition}
\newtheorem{definition}{Definition}[section]
\newtheorem{example}{Example}[section]


\begin{document}
\maketitle
\section{Abstract}

Effects are tricky, and essential!
Programmers have long sought to make them more predictable;
this may come at the cost of complexity.
In this paper I will present and explore
the relation and use of monads and algebraic effects,
I will also try to establish the theoretical foundations of these concepts.

\section{Monads}

Moggi \cite{moggi1989computational} introduced categorical semantics of computation
based on monads. Moggi defines a monad over a category $\mathcal{C}$ as a triple $(T,\eta,\mu)$,
where $T$ is a functor, $\eta$ and $\mu$ are natural transformations. He further defines a
a computational model which allows computations with effects such as non-deterministic 
computation, side effects, and continuations.

\begin{equation}
  \begin{split}
    T    &:: \mathcal{C} \rightarrow \mathcal{C} \\
    \eta &:: Id_c \rightarrow T                  \\
    \mu  &:: T^2 \rightarrow T
  \end{split}
\end{equation}

Typically a programmer will think of a monad as three function signatures
which correspond to the triple defined by Moggi; of course much more
informally, \texttt{bind} coresponds to the functor $T$, \texttt{unit} and \texttt{join} to the
natural transformations $\eta$ and $\mu$ respectively.

\begin{equation}
  \begin{split}
    bind &:: M x \rightarrow (x \rightarrow M y) \rightarrow M y \\
    unit &:: x \rightarrow M x                                   \\
    join &:: M (M x) \rightarrow M x
  \end{split}
\end{equation}

One may also see \texttt{map} which is similar to \texttt{bind}.
\begin{equation}
  map :: M x \rightarrow (x \rightarrow y) \rightarrow M y
\end{equation}

Both map and bind are typically used as infix operators.
We shall use $\gg$ as a binary operator for bind.

To ensure we are indeed talking about monads, these functions must follow a few
rules or laws. In practise these laws must be obeyed so we can chain computations
in predictable and understandable ways.

\begin{itemize}
	\item Left identity - $(unit\ a) \gg f = f a$
	\item Right identity - $m \gg unit = m$
	\item Associativity - $(m \gg f) \gg g = m \gg (\lambda x \rightarrow f x \gg g)$
\end{itemize}

%%TODO Kleisli composition operator
%%Note that these axioms mirror those of a monoid $(M,\ \cdot,\ e)$,
%%where \textit{bind} corresponds to $\cdot$ and
%%$e$ corresponds to \textit{unit} such that
%%$$\forall\ x, y, z \in M\ (x \cdot y) \cdot z = x \cdot (y \cdot z)$$
%%$$\forall\ m \in M\ m \cdot e = e \cdot m = m$$

\subsection{Use of monads}

Wadler \cite{wadler1990comprehending} demonstrated the use of monads in a pure
functional programming language to simulate effects. I will explore examples similar to
the exception monads Wadler presents.

Firstly I will attempt to demonstrate how monads are used to structure programs.
In this example we will not that OCaml has non-nullable types
i.e; one will never see null where one is expecting an
int or a binary tree or anything else.
Null values are always explicit.
The cannocical representation of null is the None variant.
Monads can be used to succintly and expressively structure computation with the option type.

\begin{verbatim}
  type 'a option = Some 'a | None
  val unit : 'a -> 'a option
  val join : 'a option option -> 'a option
  val map  : 'a option -> ('a -> 'b) -> 'b option
\end{verbatim}

The code for these three functions is simple and fairly self-evident.
However from this simple code we can construct much more complicated code which we
will be certain will never have a \textit{NullPointerException}.
Personally, such an assurance is invaluable in creating correct programs.
Here it's important to note that the effect is the \textit{NullPointerException};
we deal with it much more cleanly and effectively 100\% of the time using monads.

\begin{verbatim}
  let unit a = Some a
  let join = function
      | Some (Some a) -> Some a
      | _ -> None
  let map a f = match a with
      | None -> None
      | Some b -> f b
\end{verbatim}

Consider this example code for searching a trie data structure. Briefly,
a trie is key value data structure; where the key is a string (considered a
list of chars in this example). Each node has an option value and a list of
children. The root of the trie represents the empty string. Two auxillary
functions find child and val extract are used in the search code;
val extract simply returns the first value in a pair, find child searchs
the list of children returning the node with the matching character given
if the child exists.

\begin{verbatim}
  type ('c, 'v) t = Trie of 'v option * (('c * ('c, 'v) t) list)
  val find_child  : ('c, 'v) t -> 'c -> ('c, 'v) t option
  val val_extract : ('c, 'v) t -> 'v option
  val create      : 'c list -> 'v -> ('c, 'v) t
  val get         : ('c, 'v) t -> 'c list -> 'v option

  let create key data =
    let rec aux = function
      | []      -> Trie (Some data, [])
      | c :: cs -> Trie (None, [(c, aux cs)])
    in aux key
\end{verbatim}

Create is shown to just illustrate the data structure,
we simply iterated across the list creating a node in the trie for each character
until the list is exhausted at which point we insert the value.

It is convention to use bind as the infix operator "\texttt{>>=}".

In the search code t is the trie, key is a list of characters over which
we iterate the search function. For each character we call find child on
the current node. Find child, if succesful, will return the next node
upon which we continue the search with the remaining characters. Once we
are at the last character we know to try and extract the value from the end
node. The key point here is that using bind allows us to succintly only code
for the happy case but deal with the error case at every step.

\begin{verbatim}
  let get t key =
    let rec search chars t =
        match chars with
        | []      -> val_extract t
        | c :: cs -> bind_search (find_child t c) cs
    and bind_search ot chars = ot >>= search chars
    in search key t
\end{verbatim}

The result type is similar to the option type, however we use an extra type
parameter for the unhappy case, essentially the result type encapsulates 
either an error or a correct computation result. The result monad
corresponds to a generalisation of the exception monad presented by Wadler \cite{wadler1995monads}.

\begin{verbatim}
  type ('a, 'e) result = Ok of 'a | Error of 'e
  val bind : ('a, 'e) result
          -> ('a -> ('b, 'e) result)
          -> ('b, 'e) result
\end{verbatim}

In this example we have a list assocation which is a list of pairs, in this case both
items in the pairs are strings.

\begin{verbatim}
  let list_assoc_to_job la =
    let find k = match List.Assoc.find la k with
      | None -> Error (Err.missing_key k)
      | Some v -> Ok v
    in
    find "name"   >>=                  (fun name   ->
    find "prog"   >>=                  (fun prog   ->
    find "args"   >>= parse_args   >>= (fun args   ->
    find "run_at" >>= parse_run_at >>= (fun run_at ->
      Ok (Job.create name prog args run_at ())
    ))))
\end{verbatim}

Bind is used to succintly short circuit a computation when a value can not be
correctly obtained. As such this allows the program to be structured neatly to return
an error with precise information for which key could not be found or which value could
not be parsed. The result monad is very similar to the option monad. It would be interesting
to examine whether algebraic effects can be used to structure a program in a similar manner.

Typically programming languages allow the elison of the anonymous function on the right hand side
of the bind to simply introduce the binding into the environment;
in this case OCaml ppx extension points are used in the form \textit{let\%bind}.
This is largely just a programmer convienence,
for comparison Haskell offers \textit{do notation} for the same end.

\begin{verbatim}
  let list_assoc_to_job la =
    let find k = match List.Assoc.find la k with
      | None -> Error (Err.missing_key k)
      | Some v -> Ok v
    in
    let%bind name   =  find "name"                     in
    let%bind prog   =  find "prog"                     in
    let%bind args   = (find "args"   >>= parse_args  ) in
    let%bind run_at = (find "run_at" >>= parse_run_at) in
    Ok (Job.create name prog args run_at ())
\end{verbatim}

%%TODO
Imperative programming is undeniably popular, and much closer to how computers physically work,
monads have been concretely shown to be an effective method of replicating
imperative style and retaining benefits of purity \cite{PeytonJones:1993}.
These examples clearly illustrate how useful monads can be for structuring a program,
however the importance and necessity of monads is that for pure functional languages
to be actually useful we require effects. Monads give us means to achieve that.

\section{Algebraic Effects}

%%TODO Allow us to reconcile being with doing
%%allowing the programmer to separate the expression of an effectful computation from its implementation."\cite{dolan2015effective}

Algebraic effects, like monads, can be used to model effects in a pure language.
Where Moggi \cite{moggi1989computational} proposed monads to give categorical semantics to computational effects;
Power and Plotkin \cite{Plotkin:2002dw} propose "computational effects as being realised by
families of operations, with a monad being generated by their equational theory".
This means we can treat effects algebraically,
to actually interact with them as programming constructs algebraic effects are paired with
handlers \cite{plotkin2009handlers}.
Where algebraic operations construct effects handlers are the dual, they deconstruct effects.

Currently algebraic effects are a popular area of research,
it has been shown that they can "concisely describe many complex control-flow constructs" \cite{leijen2017type}.
They "provide a modular abstraction for expressing effectful computation"\cite{dolan2015effective},
in this section I will examine to what extent they can be equivalently used to monads and wether
they provide the same benefits.

%%"the signature of the effect operations forms a free algebra which gives rise to a free monad. Free monads provide a natural way to give semantics to effects, where handlers describe a fold over the algebra of operations [44]. Using a more operational perspective, we can also view algebraic effects as resumable exceptions (or perhaps as a more structured form of delimited continuations)." \cite{leijen2017type}


\medskip

\bibliographystyle{unsrt}
\bibliography{dissert}

\end{document}
