\section{Categorical Views of Effects}

\subsection{Overview}
In this section we will explore the theoretical underpinnings
of monads and algebraic effects.
In the 1960s category theorists constructed monads
Category theorists invented monads in the 1960s to concisely of universal algebra

the study of algebraic theories and their models or algebras. 

\cite{wadler1990comprehending}
\cite{hyland2007category}


From monads and Lawvere theories we have
the $\lambda_C$-calculus and algebraic effects.

Briefly,

From Moggi’s assignment to each set X of the set TX of values associated with a computational effect,
one has passed to the study of the operations associated with the computational effect
\cite{hyland2007category}

A key understanding is that operations generate monads
notions of computation determine monads
\cite{plotkin2001adequacy}

Plotkin and Power verified that and, later joined by Hyland,
saw that what were then regarded as computational effects could,
with one exception, fruitfully be seen as an instance and a development of universal algebra:
interactive input/output was soon recognised as an example involving no equations;
it was seen how to incorporate local state naturally; and the various ways of combining computational effects proved to be simple instances of combining Lawvere theories.
\cite{hyland2007category}


\subsection{Categories}
First we must introduce the concept of \textit{categories},

\begin{definition}
    A \textit{category} $C$ consists of
    \begin{itemize}
        \item A set $Ob\,C$, elements of which are called \textit{objects} of $C$.
        \item For each $X, Y \in Ob\,C$
            a set $Hom(X,Y)$ called the \textit{homset} from $X$ to $Y$.
        \item A \textit{composition function} $\circ$ such that
            \begin{equation}
                \circ : Hom(X,Y) \times Hom(Y,Z) \rightarrow Hom(X,Z)
            \end{equation}
        \item For all $X \in Ob\,C$ an element $Id_X$ of $Hom(X,\ X)$ such that
            \begin{equation}
            f \circ Id_X = Id_Y \circ f = f
            \end{equation}
        \item Composition is associative.  $f \circ (g \circ h) = (f \circ g) \circ h$
    \end{itemize}
\end{definition}

\par
An element $f$ of $Hom(X,Y)$ is called an \textit{arrow},
or a \textit{morphism}. The object $X$ is called the \textit{domain} of $f$ and $Y$ is
the \textit{codomain}.\\

\begin{example}
    The immediate example of a category is the category of sets.
    The objects in $C$ are (small) sets,
    a morphism from $X$ to $Y$ is a function $f : X \rightarrow Y$.
    The composition of Set is given by composition of functions,
    and the identity maps are given by the identity functions.
\end{example}

\subsection{Functors}
\begin{definition}
    A \textit{functor} $U : C \rightarrow D$ consists of
    \begin{itemize}
        \item A function $Ob\,U : Ob\,C \rightarrow Ob\,D$.
        \item A function $U : Hom_C(X,Y) \rightarrow Hom_D(UX, UY)$
            such that $U$ respects both composition and identity.
            I.e.
            \begin{equation}
                Uf \circ Ug = U(f \circ g)
            \end{equation}
            \begin{equation}
                U\,Id_X = Id_{UX}
            \end{equation}
    \end{itemize}
\end{definition}

\begin{definition}
    An \textit{endofunctor} is a functor $U : C \rightarrow C$;
    i.e. the domain and codomain of the functor are the same category $C$.
\end{definition}

\subsection{Natural transformations}
\begin{definition}
    Given categories $C$ and $D$,
    with functors $U, V : C \rightarrow D$
    a \textit{natural transformation} $\alpha : U \rightarrow V$
    consists of
    \begin{equation}
        \forall\ X \in Ob\,C\ \textrm{a map} \ \alpha_X : UX \rightarrow VX
    \end{equation}
    such that $\forall\ f : X \rightarrow Y$ the following commutes
    \begin{center}
        \begin{tikzcd}[sep=large]
            UX \rar{\alpha_X} \dar[swap]{Uf} & VX \dar{Vf} \\
            UY \rar{\alpha_Y}                & VY
        \end{tikzcd}
    \end{center}
    A natural transformation can be considered a morphism of functors.
\end{definition}

\subsection{Monads}
"Monads are among the most pervasive structures in category theory
and its applications: for example, they are central to the category-theoretic account of universal algebra"
\cite{mac2013categories}
\begin{definition}
    A \textit{monad} is defined as the triple
    \begin{itemize}
        \item An endofunctor $T : C \rightarrow C$
        \item A natural transformation $\eta : 1_{C} \rightarrow T$
        \item A natural transformation $\mu : T^2 \rightarrow T$
    \end{itemize}
    Such that the following diagrams commute
    \begin{center}
        \begin{tikzcd}[sep=large]
            T \rar{\eta_T} \drar[swap]{1_{C}} & T^2 \dar{\mu} & \lar[swap]{T\eta} \dlar{1_{C}} T \\
                                               & T            &
        \end{tikzcd}
        \quad
        \begin{tikzcd}[sep=large]
            T^3 \rar{T\mu} \dar[swap]{\mu_T} & T^2 \dar{\mu} \\
            T^2 \rar[swap]{\mu}                    & T
        \end{tikzcd}
    \end{center}
\end{definition}
%TODO
%which satisfies also an extra equalizing requirement: ηA: A → T A is an equalizer of ηTA and T(ηA), i.e. for any f:B → TA s.t. f;ηTA = f;T(ηA) there exists a unique m:B → A s.t. f = m;ηA3.

\subsection{Kleisli Category}
\begin{definition}
    Given a monad $(T,\eta,\mu)$ over a category $C$,
    the \textit{Kleisli category} $C_T$ consists of
    \begin{itemize}
        \item Objects of $C_T$ are objects from the underlying category $C$.
        \item $Hom_{C_T}(X,Y) = Hom_C (X,TY)$
        \item Identity morphisms in $C_T$ are $\eta$ in $C$
        \item Composition $f \circ g$ is $\mu(Tf)g$
    \end{itemize}
\end{definition}

Composition in $C_T$ can be described in more detail via the operator
\begin{equation}
    (-)^{*} : Hom(X, TY) \rightarrow Hom(TX, TY)
\end{equation}
where given a morphism $f: X \rightarrow TY$ we have
\begin{equation}
    f^{*} = \mu_{Y} \circ Tf
\end{equation}
Note that
\begin{equation}
    \eta_{X}^{*} = Id_{TX}
    \quad\textrm{and}\quad
    f^{*} \circ \eta _{X} = f
\end{equation}
Then we can define the \textit{Kleisli operator} $\gg$ where
\begin{equation}
    g \gg f = g^{*} \circ f
\end{equation}
\begin{equation}
    x
    \stackrel{f}{\rightarrow}     T y
    \stackrel{T g}{\rightarrow}   T T z
    \stackrel{\mu z}{\rightarrow} T z
\end{equation}
where $\gg$ has these axioms
\begin{equation}
    (f \gg g) \gg h \equiv f \gg (g \gg h)
\end{equation}
\begin{equation}
    \eta_Y \gg f \equiv f \equiv f \gg \eta_X
\end{equation}

\subsection{Computational Lambda Calculus}
Moggi \cite{moggi1989computational}
introduced categorical semantics for computation based on monads.
He extended the simply typed lambda-calculus to
the computational lambda-calculus, or $\lambda_c$-calculus,
which allows computations with effects such as
non-determinism, side effects, and continuations.
In this model "a program denotes a morphism from $A$
(the object of values of type $A$) to $TB$
(the object of computations of type B)"
for example
"partial computations (of type $B$) is the lifting $B + \{\bot\}$".
Another example would be the type \texttt{IO Int}
meaning interactive input/output computations of integers.
Whereas the simply typed lambda calculus is modeled by a cartesian closed category $C$;
a $\lambda_c$-model over a category $C$ with finite products is a strong monad $(T,\eta,\mu,t)$
together with a $T$-exponential for every pair $\langle A, B\rangle$ of objects in $C$
\cite{moggi1989computational}.

Here are two examples of the $\lambda_C$-calculus from
\cite{moggi1989computational}\cite{moggi1991notions}.
\vspace{5mm}

\begin{example}\label{lc1}
\end{example}
    Computations with side-effects:
    \begin{itemize}
        \item $T(-)$ is the functor $(-\times S)^S$, where $S$ is a nonempty set of stores.
            Intuitively a computation takes a store and returns a value together with the modified store.
        \item $\eta_A$ is the map $a \rightarrow (\lambda s:S.\langle a,s \rangle)$
        \item $\mu_A$ is the map $f \rightarrow (\lambda s:S.eval(fs))$,
            i.e. $\mu_A(f)$ is the computation that given a store $s$,
            first computes the pair computation-store $\langle f\prime,s\prime\rangle = fs$
            and then returns the pair value-store $\langle a,s\prime\prime\rangle = f\prime s\prime$.
    \end{itemize}
\vspace{5mm}

\begin{example}\label{lc2}
    Non-deterministic computations:
    \begin{itemize}
        \item $T(-)$ is the covariant powerset functor,
            i.e.  $T(A)$ = $P(A)$ and $T(f)(X)$ is the image of X along f
        \item $\eta_A$ is the singleton map $a \mapsto  \{a\}$
        \item $\mu_A(X)$ is the big union $\bigcup X$
    \end{itemize}
\end{example}

\subsection{Lawvere Theories}
The semantics of algebraic effects have a categorical basis as
"countable enriched Lawvere theories freely generated by $\dots$ operations and equations".
\cite{plotkin2004computational}
"In mathematical practice Lawvere theories arise
whenever one has a functor into a category with finite products
and one studies the natural transformations between finite products of the functor".
\cite{hyland2007category}
A more in-depth mathematical background of Lawvere theories is, regrettably,
beyond the scope of this document.

The essential semantic difference between monads and algebraic effects is that
where monads have a constructed object $TX$,
i.e. the computations of the type $X$,
algebraic effects have operations from which effects arise;
and thus $TX$ is derived not constructed.
Consider the monad $IO Int$ which is the object of computations
of the type $Int$;
from an algebraic effect perspective
one has the operation \texttt{print} which generates the effect

An example of the Lawvere Theory $L_{I/O}$ for input/output is \textit{generated} by the operations
\begin{equation}
read : I \rightarrow 1 \quad\textrm{and}\quad write : 1 \rightarrow O
\end{equation}
where $I$ is a countable set of inputs and $O$ of outputs.

\cite{plotkin2001adequacy}
interactive input/output is more directly modelled
by the Lawvere theory $L_{I/O}$ than by the corresponding monad
\begin{equation}
TX = \mu Y.(O \times Y + Y^I + X)
\end{equation}

An enriched Lawvere theory is generated by operations subject to equations.
Operations appear directly in describing programming languages,
but equations do not.
One of the equations for side-effects is
\begin{equation}
    updateloc,v (updateloc,v\prime (x)) = updateloc,v\prime (x)
\end{equation}
and the corresponding program assertion is
\begin{equation}
(l := x;let y be !l in M) = (l := x;M[x/y])
\end{equation}
This example is from \cite{plotkin2001adequacy}.

A potential advantage of algebraic effects over monads is that
the "
notion of countable enriched Lawvere theory
provides us with a natural way to describe
how computational effects may be combined.
"
\cite{plotkin2004computational}.
I.e. algebraic effects compose more naturally.


A further advantage may be that
"a class of theories that can be viewed as categories with a monad,
so that
any category with a monad is, up to equivalence (of categories with a monad), one of such theories.
Such a reformulation in terms of theories is more suitable for formal manipulation and more appealing to those unfamiliar with Category Theory."
\cite{moggi1991notions}


The Lawvere theory LE for exceptions is the free Lawvere theory generated by E operations $raise : 0 \rightarrow  1$,
where E is a set of exceptions.
In terms of operations and equations,
this corresponds to an E-indexed family of nullary operations with no equations.
%The monad on Set generated by LE is TE = − + E. More generally, the forgetful functor UL : Mod(LE,C) C induces the monad
%−+E on C, where E is the E-fold copower of 1, i.e.,  exists in C.
It is shown in [39]
that this countable Lawvere theory induces Moggi’s side-effects monad $(S \times \textrm{-})^S$ on
Set.
\cite{hyland2007category}
