\section{Algebraic Effects}
Following from our discussion of Lawvere theories

the essence is we have operations which generate effects,
which can be dealt with algebraically %(What does algebraic mean
and further which allows generic effects




"A sloppy but intuitive way to describe this is "algebraic effects operate algebraically, which is to say, subject to purely local rewrites, or, alternately, in such a way that the interpretation of an expression can be given as an operator on the interpretation of its subexpressions." Continuations are nonlocal, and can cause expressions to need global knowledge for the purposes of interpretation."
cite internet
Another, even more handwavy way to think of this, is that handlers of algebraic effects are given via continuations. So now, analogously to what happens when one tries to combine continuations with exceptions in scheme, we run into the age-old conflict -- two mechanisms that both affect global control flow exist at once, and have no particularly "good" way to not step on one another's toes.



Currently algebraic effects are a popular area of research,

\textbf{Some Algebraic Effect Implementations}
\begin{itemize}
    \item Eff: Matija Pretnar and Andrej Bauer, 2011-now.
    \item Links : Daniel Hillerström and Sam Lindley, 2015.
    \item Frank: Conor McBride, 2007, 2012.
    \item OCaml+effects: Stephen Dolan, Leo White, KC Sivaramakrishnan, 2016.
\end{itemize}

\subsection{Overview}
Algebraic effects, like monads, can be used to model effects in a pure language.
Where Moggi \cite{moggi1989computational} proposed monads to give categorical semantics to computational effects;
Power and Plotkin \cite{Plotkin:2002dw} propose "computational effects as being realised by
families of operations, with a monad being generated by their equational theory".
This means we can treat effects algebraically,
to actually interact with them as programming constructs algebraic effects are paired with
handlers \cite{plotkin2009handlers}.
Where algebraic operations construct effects handlers are the dual, they deconstruct effects.

\begin{example}
    Consider the examples of effects we gave in cite
    we will now give the operations which generate those effects
    \begin{itemize}
        \item Input/Output - print
        \item Mutable state - get/set
        \item Exceptions - raise
        \item Non-Determinism and Probabilistic Non-Determinism - choose
        \item Continuations - cont
    \end{itemize}
\end{example}

\begin{definition}
    An operation is !

    \begin{verbatim}
        effect IO =
          | Print : string -> unit
          | Read : string

        perform (Print "Hello world");
    \end{verbatim}
\end{definition}

\begin{verbatim}
effect choice =
  | Choose : bool

effect IO =
  | Print : string -> unit
  | Read : string

effect int_state =
  | Get : int
  | Set : int -> unit

effect scheduler =
  | Spawn : (unit -> unit) -> unit
  | Yield : unit

let incr_twice () : int =
  perform (Set ((perform Get) + 1));
  perform (Set ((perform Get) + 1));
  perform (Print "incremented twice");
  perform Get
\end{verbatim}
%http://gallium.inria.fr/~scherer/doc/effect-handlers-talk.html#/sec-effect-handlers--examples

\subsection{Handlers}
\cite{Plotkin:2002dw}

\subsection{Draft}

"Of the various operations, handle is of a different computational character and, although natural, it is not algebraic
Andrzej Filinski (personal communication) describes handle as a deconstructor, whereas the other operations are constructors (of effects). In this paper, we make the notion of constructor precise by identifying it with the notion of algebraic operation."

it has been shown that they can "concisely describe many complex control-flow constructs" \cite{leijen2017type}.
They "provide a modular abstraction for expressing effectful computation"\cite{dolan2015effective},
in this section I will examine to what extent they can be equivalently used to monads and wether
they provide the same benefits.





%"There are computationally natural families of operations associated with sev- eral of the above monads: one has ‘raise’ operations for exceptions; one has ‘read’ and ‘write’ operations associated with interactive input/output; one has a nonde- terministic binary ‘choice’ operation when modelling nondeterminism; one has a ‘random choice’ operation for probabilistic nondeterminism; and one has ‘lookup’ and ‘update’ operations when modelling global state. An analysis of several of thesefamiliesofoperationsappearsin[20]:theyareregardedasalgebraicfami- lies of operations associated with an already given monad, and are characterised intermsofgenericeffects:e.g.,togiveagenericeffecte:n−→Tmisequivalent togivingmn-aryalgebraicfamiliesofoperations,wheremandnneednotbe finite(misthem-foldcoproductof1inC).Cruciallywhenamonadisgiven byalgebraicoperationsandequationsinthesenseof[10],thealgebraicfamilies of operations associated with it are given by the derived operations."
\cite{Plotkin:2002dw}

It is important to seperate values from operations, operations give rise to effects,
there is a difference here in being and doing

"As a restriction on general monads, algebraic effects come with various advantages:
they can be freely composed, 
and there is a natural separation between their interface (as a set of operations) and their semantics (as a handler)."

%%"the signature of the effect operations forms a free algebra which gives rise to a free monad. Free monads provide a natural way to give semantics to effects, where handlers describe a fold over the algebra of operations [44]. Using a more operational perspective, we can also view algebraic effects as resumable exceptions (or perhaps as a more structured form of delimited continuations)." \cite{leijen2017type}

Monad transformers can quickly become unwieldy when there are lots of effects to manage, leading to a temptation in larger programs to combine everything into one coarse-grained state and exception monad.

[Programming and Reasoning with Algebraic Effects and Dependent Types]

%%TODO Allow us to reconcile being with doing
%%allowing the programmer to separate the expression of an effectful computation from its implementation."\cite{dolan2015effective}

Algebraic effects are computational effects that can be represented by an equational theory, or algebraic theory, whose operations produce the effects at hand.
Continuations are not algebraic effects.

Algebraic effects allow computational effects to be representable by
effects that allow a representation by opera- tions and equations

\subsection{Overview}
\subsection{Handlers}
\subsection{Rebuilding our trie with algebraic effects}
\subsection{Composing Algebraic Effects}
\subsection{Comparison with Monads}
\subsection{Structuring Programs with Algebraic Effects}
