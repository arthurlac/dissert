\section{Algebraic Effects}
Following from our discussion of Lawvere theories
in section three,
we will now explore 
The essence is
we have operations which generate effects,
which can be dealt with algebraically
%(What does algebraic mean
"A sloppy but intuitive way to describe this is "algebraic effects operate algebraically,
which is to say,
subject to purely local rewrites, or, alternately,
in such a way that the interpretation of an expression
can be given as an operator on the interpretation of its subexpressions."
Another, even more handwavy way to think of this,
is that handlers of algebraic effects are given via continuations.
So now, analogously to what happens when one tries to combine continuations
with exceptions in scheme, we run into the age-old conflict --
two mechanisms that both affect global control flow exist at once,
and have no particularly "good" way to not step on one another's toes.
cite internet

\subsection{Overview}


allowing the programmer to separate the expression of an effectful computation from its implementation."
\cite{dolan2015effective}
This is a benefit we saw with monads,
seperating the happy path from our error handling

Currently algebraic effects are a popular area of research,
and may well enter the mainstream in the next decade;
potentially because imperative programmers
will find them more intuitive equivalent monads.

AE represent the cutting edge

\textbf{Some Algebraic Effect Implementations}
\begin{itemize}
    \item Eff: Matija Pretnar and Andrej Bauer, 2011-now.
    \item Links : Daniel Hillerström and Sam Lindley, 2015.
    \item Frank: Conor McBride, 2007, 2012.
    \item OCaml+effects: Stephen Dolan, Leo White, KC Sivaramakrishnan, 2016.
\end{itemize}

Algebraic effects, like monads, can be used to model effects in a pure language.
Where Moggi \cite{moggi1989computational} proposed monads to give categorical semantics to computational effects;
Power and Plotkin \cite{Plotkin:2002dw} propose "computational effects as being realised by
families of operations, with a monad being generated by their equational theory".
This means we can treat effects algebraically,
to actually interact with them as programming constructs algebraic effects are paired with
handlers \cite{plotkin2009handlers}.
Where algebraic operations construct effects handlers are the dual, they deconstruct effects.
"As a restriction on general monads, algebraic effects come with various advantages:
they can be freely composed,
\cite{leijen2017type}
They "provide a modular abstraction for expressing effectful computation"\cite{dolan2015effective},


\begin{example}
    Consider the examples of effects we gave at the beginning of this document.
    Each of these effects has a natural operation which generates the effect.
    Consider
    \begin{itemize}
        \item Input/Output \textit{generated by} \texttt{read, write}
        \item Mutable state \textit{generated by} \texttt{set, get}
        \item Exceptions \textit{generated by} \texttt{raise}
        \item Non-Determinism \textit{generated by the binary} \texttt{choose}
        \item Probabilistic Non-Determinism \textit{generated by} \texttt{random choice}
    \end{itemize}
\end{example}

NOT CONTINUATIONS

"There are computationally natural families of operations associated with several of the above monads:
An analysis of several of thesefamiliesofoperationsappearsin[20]:
theyareregardedasalgebraicfami- lies of operations associated with an already given monad,
and are characterised intermsofgenericeffects:e.g.,
togiveagenericeffecte:nTmisequivalent togivingmn-aryalgebraicfamiliesofoperations,
wheremandnneednotbe finite(misthem-foldcoproductof1inC).
Cruciallywhenamonadisgiven byalgebraicoperationsandequationsinthesenseof,
thealgebraicfamilies of operations associated with it are given by the derived operations."
\cite{Plotkin:2002dw}

\begin{definition}
    an operation is !
\end{definition}

\begin{definition}
    A \textit{generic effect} is
\end{definition}

\begin{example}
    Consider this example of an IO effect,
    first we define an effect \texttt{IO}
    and then describe two constructors
    \texttt{Print} and \texttt{Read}.
    To actually have an effect we use the primitive
    perform on an effect we construct, in this case with \texttt{Print}.
    \begin{verbatim}
        effect IO =
          | Print : string -> unit
          | Read : string

        perform (Print "Hello world");
    \end{verbatim}
\end{example}

\subsection{Handlers}
"Of the various operations, handle is of a different computational character and, although natural, it is not algebraic
Andrzej Filinski (personal communication) describes handle as a deconstructor, whereas the other operations are constructors (of effects). In this paper, we make the notion of constructor precise by identifying it with the notion of algebraic operation."





\cite{Plotkin:2002dw}





and there is a natural separation between
their interface (as a set of operations)
and
their semantics (as a handler)."
"the signature of the effect operations forms a free algebra which gives rise to a free monad.
Free monads provide a natural way to give semantics to effects,
where handlers describe a fold over the algebra of operations.
Using a more operational perspective,
we can also view algebraic effects as resumable exceptions
(or perhaps as a more structured form of delimited continuations)."
\cite{leijen2017type}

Monad transformers can quickly become unwieldy when there are lots of effects to manage,
leading to a temptation in larger programs to combine everything into one coarse-grained state and exception monad.
[Programming and Reasoning with Algebraic Effects and Dependent Types]

Algebraic effects are computational effects that can be represented by an equational theory, or algebraic theory, whose operations produce the effects at hand.
Continuations are not algebraic effects.

Algebraic effects allow computational effects to be representable by
effects that allow a representation by opera- tions and equations

\subsection{Rebuilding our trie with algebraic effects}
\begin{verbatim}




  let get t key =
    let rec search chars t =
        match chars with
        | []      -> val_extract t
        | c :: cs -> bind_search (find_child t c) cs
    and bind_search ot chars = ot >>= search chars
    in search key t



\end{verbatim}



\begin{verbatim}
class Monad m => MonadState s m | m -> s where
    -- | Return the state from the internals of the monad.
    get :: m s
    get = state (\s -> (s, s))

    -- | Replace the state inside the monad.
    put :: s -> m ()
    put s = state (\_ -> ((), s))

    -- | Embed a simple state action into the monad.
    state :: (s -> (a, s)) -> m a
    state f = do
      s <- get
      let ~(a, s') = f s
      put s'
      return a

join :: (State s (State s a)) -> (State s a)
join xss = State (\s -> uncurry runState (runState xss s))

--------------------------------------------------------------------
type GameValue = Int
type GameState = (Bool, Int)

playGame :: String -> State GameState GameValue
playGame []     = do
    (_, score) <- get
    return score

playGame (x:xs) = do
    (on, score) <- get
    case x of
         'a' | on -> put (on, score + 1)
         'b' | on -> put (on, score - 1)
         'c'      -> put (not on, score)
         _        -> put (on, score)
    playGame xs

startState = (False, 0)

main = print $ evalState (playGame "abcaaacbbcabbab") startState
\end{verbatim}

\subsection{Structuring Programs with Algebraic Effects}
As we demonstrated monads are rather useful for structuring programs,
allowing us to only consider the happy path while remaining assured
edge cases are handled;
it is natural to ask if programming with algebraic effects
has the same properties?

\begin{example}
\end{example}

\subsection{Composing Algebraic Effects}
What has become the most evident deficiency of monads is
how difficult it is to compose them.
Composability is often touted as one the most
prominent advantages of functional programming

How to compose algebraic effects?
\begin{example}
\end{example}

\subsection{Comparison with Monads}
lots of ppl say the monads when composed make effects coarse grained
and this is ugly


algebraic effects are more scalable solution for engineering programs!
\begin{verbatim}
effect choice =
  | Choose : bool

effect int_state =
  | Get : int
  | Set : int -> unit

effect scheduler =
  | Spawn : (unit -> unit) -> unit
  | Yield : unit

let incr_twice () : int =
  perform (Set ((perform Get) + 1));
  perform (Set ((perform Get) + 1));
  perform (Print "incremented twice");
  perform Get
\end{verbatim}
%http://gallium.inria.fr/~scherer/doc/effect-handlers-talk.html#/sec-effect-handlers--examples
