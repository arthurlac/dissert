\documentclass[a4paper,10pt]{article}
\usepackage{amsmath}

\title{A comparison of algebraic effects and monads}
\author{Arthur}
\date{\today}

\setlength{\parindent}{1em}
\setlength{\parskip}{1em}

\begin{document}
\maketitle
\section{Introduction}

In the dissertation I will explore both the relation and use of monads and algebraic effects,
I will also try to establish the theoretical foundations of these concepts.
Finally I will present programs where monads and algebraic effects are equivalent or
inequivalent. Concluding with a discussion of the ramifications of the latter.

\subsection{Topic overview}
Moggi \cite{Moggi:hc} introduced the computational lambda calculus (${\lambda}_c$-calculus)
and computational monads. Demonstrating non-determinism, side-effects, and continuations.

In pure function functional languages monads have become the de facto abstraction for effects.
This practise was established in this paper \cite{wadler1990}.

Algebraic effects \cite{plotkin2001adequacy} and their handlers \cite{Plotkin:2001jr}
can be used for non-determinism, side-effects, and continuations just like monads.
Algebraic effects have seen a rise in popularity in functional programming languages.
With effect handlers programs can be written in an imperative direct style.
In my dissertation I will try to ascertain the state of the art in this area.
See \cite{Bauer:2013fn, leijen:16, Lindley:2016vz, Dolan:2017} and more.



\section{Plan}

\begin{enumerate}
  \item Comprehending monads                     \cite{wadler1990}
  \item Computational lambda-calculus and monads \cite{Moggi:hc}
  \item Semantics for Algebraic Operations       \cite{Plotkin:2001jr}
\end{enumerate}

of establishing and analysing the theoretical foundations of both monads and algebraic effects.




I will also read papers in that area to develop a more complete understanding
\cite{wadler2003marriage, wadler1994monads, PeytonJones:1993}.
Perhaps reflecting on my usage of monads in programming and developing some sample
programs to compare to algebraic effects further on.

A monad over a category $\mathcal{C}$ is a triple $(T,\eta,\mu)$. Where

\begin{equation}
  \begin{split}
    T    &:: \mathcal{C} \rightarrow \mathcal{C} \\
    \eta &:: Id_{\mathcal{C}} \rightarrow T       \\
    \mu  &:: T^{2} \rightarrow T
  \end{split}
\end{equation}
and the following holds
\begin{equation}
  \begin{split}
    \eta &:: Id_{\mathcal{C}} \rightarrow T       \\
    \mu  &:: T^{2} \rightarrow T
  \end{split}
\end{equation}


Typically a program will think

\begin{equation}
  \begin{split}
    return &:: x \rightarrow M x                                 \\
    map    &:: (x \rightarrow y) \rightarrow M x \rightarrow M y \\
    join   &:: M (M x) \rightarrow M x
  \end{split}
\end{equation}

One may also see

\begin{equation}
  bind :: M x \rightarrow (x \rightarrow M y) \rightarrow M y
\end{equation}






Next I will focus on \cite{Moggi:hc}.
Finally \cite{Plotkin:2001jr} also reading \cite{plotkin2001adequacy, Plotkin:2002dw}.
The latter two papers require the most theoretical knowledge, a considerable portion
of time must be allocated for this.
I will be focusing on reading primarily \cite{barr1990category} for this background knowledge.

The natural conclusion of my literature survey should be the resolution to the goals I set out
in the beginning of this document; to "explore both the relation and use of monads and
algebraic effects".
The success of my literature survey could be judged by how effectively these goals are 
achieved.

What follows my literature survey should be first to
  "present programs where monads and algebraic effects are equivalent or inequivalent"
and then
  "concluding with a discussion of the ramifications of the latter".
After the literature survey is complete I will of course be in a better position to decide
the best way to achieve these goals, thus towards the end of my literature survey I will
review my plan for those goals.


\medskip

\bibliographystyle{unsrt}
\bibliography{proposal}

\end{document}
