\documentclass[a4paper,10pt]{article}
\usepackage{amsmath}

\title{Proposal}
\author{Arthur}

\begin{document}
\section{Introduction}

In the dissertation I will explore both the relation and use of monads and algebraic effects.
I will also try to establish the theoretical foundations of these concepts.
Finally I will present programs where monads and algebraic effects are equivalent or
inequivalent. Concluding with a discussion of the ramifications of the latter.

\subsection{Topic overview}
Moggi \cite{Moggi:hc} introduced the computational lambda calculus (${\lambda}_c$-calculus)
and computational monads. Demonstrating non-determinism, side-effects, and continuations.

\par
In pure function functional languages monads have become the de facto abstraction for effects.
This practise was established \cite{wadler1990}.

\par
Algebraic effects \cite{plotkin2001adequacy} and their handlers \cite{Plotkin:2001jr}
can be used for non-determinism, side-effects, and continuations just like monads.

\par
Along with handlers \cite{Plotkin:2009dr},
algebraic effects have seen a rise in popularity in functional programming languages.
With effect handlers programs can be written in an imperative direct style.
In my literature survey I will try to ascertain the state of the art in this area.
See \cite{Lindley:2016vz, Dolan:2017, leijen:16, Bauer:2013fn} and more.



\section{Plan}

\begin{enumerate}
  \item Literature survey, weeks 4 - 11, 19. Focusing on:
    \begin{enumerate}
      \item Comprehending monads                     \cite{wadler1990}
      \item Computational lambda-calculus and monads \cite{Moggi:hc}
      \item Semantics for Algebraic Operations       \cite{Plotkin:2001jr}
    \end{enumerate}
  \item Demonstration of progress week 21
  \item Final deadline week 31
\end{enumerate}

Firstly there is the literature survey where I will be able to achieve my first objective
of establishing and analysing the theoretical foundations of both monads and algebraic
effects.
I will do this by focusing on the analysis and understanding of three key papers.
This will necessitate the understanding of auxiliary and related papers.

\par
For the literature survey a deadline of week eight is suggested however for this project I would
counter suggest a deadline of week nineteen. Simply because the understanding of the theoretical
foundations is crucial to the success of the dissertation.
Of course these papers require considerable background knowledge, as such
I intend to allocate three weeks each for these papers.

\par
Beginning with paper 1a;
I will also read papers in that area to develop a more complete understanding
\cite{wadler2003marriage, wadler1994monads, PeytonJones:1993}.
Next I will focus on \cite{Moggi:hc}.
Finally \cite{Plotkin:2001jr} also reading \cite{plotkin2001adequacy, Plotkin:2002dw}.

\par
The natural conclusion of my literature survey should be the resolution to the goals I set out
in the beginning of this document. Specifically
  "explore both the relation and use of monads and algebraic effects"
and
  "establish the theoretical foundations of these concepts";
the success of my literature survey could be judged by how effectively these goals are 
achieved.

\par
What follows my literature survey should be first to
  "present programs where monads and algebraic effects are equivalent or inequivalent"
and then
  "concluding with a discussion of the ramifications of the latter".
After the literature survey is complete I will of course be in a better position to decide
the best way to satisfy these goals;


\medskip

\bibliographystyle{unsrt}
\bibliography{proposal}

\end{document}
