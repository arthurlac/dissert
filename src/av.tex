\section{Abridged version}

A concrete example of fmap/bind/join expressed in JavaScript or ECMAScript is
\begin{verbatim}
    const arr = [1,2,3,4];
    const fmap = (f,x) => x.map(f); //Easy in JS for arrays
    console.log(fmap(x => x + 1, arr));
    // === Array [2,3,4,5]

    //Note we use 'unit' as return is a keyword
    const unit = x => [x];
    console.log(unit(1));
    // === Array [1]

    const arr2 = [[1, 2], [3, 4]];
    const append = (onto, val) => onto.concat(val);
    const join = x => x.reduce(append, []);
    console.log(join(arr2));
    // === Array [1, 2, 3, 4]

    const bind = (x, f) => join(fmap(f,x));
    const double = x => [x,x];
    console.log(bind(arr1, double));
    // === Array [1, 1, 2, 2, 3, 3, 4, 4]
\end{verbatim}

Or in OCaml
\begin{verbatim}
    let fmap f x =
        let rec aux = function
        | [] -> []
        | hd :: tl -> (f hd) :: (aux tl)
    in aux x

    val fmap : ('a -> 'b) -> 'a list -> 'b list

    fmap (fun x -> x + 1) [1;2;3;4];;
    - : int list = [2; 3; 4; 5]

\end{verbatim}
Then we have
\begin{verbatim}
    let return x = [x];;

    val return : 'a -> 'a list

    return 1;;
    - : int list = [1]
\end{verbatim}
Then we have
\begin{verbatim}
    let rec join = function
    | [] -> []
    | hd :: tl -> hd @ (join tl)

    val join : 'a list list -> 'a list

    join [[1;2];[3;4]];;
    - : int list = [1; 2; 3; 4]
\end{verbatim}
