\section{What are effects?}
If we are going to assert that effects are necessary
and difficult to manage we should explain why this is.
Firstly, we shall explore what effects actually are.\\

\begin{example}
    A good place to start is to give several example of effects
    \begin{itemize}
        \item Input/Output
        \item Mutable state
        \item Exceptions
        \item Non-Determinism and Probabilistic Non-Determinism
        \item Continuations
    \end{itemize}
\end{example}

Conceptually, programmers understand an \textit{effect} to be the semantics of any code
which is not entirely for deterministically calculating a result for a function.
We should note that side-effects are a subset of effects,
for example input/output and mutable states are side effects
however non-determinism is not.
The distinction being that a side effect
has an observable effect or interaction on the "outside world".\\

\begin{definition}
    A function $f$ is \textit{referentially transparent} if
    for any functon $g$ which calls $f$,
    any call of $f$ can be replaced with the value returned by $f$
    without changing the value returned by $g$.

    An obvious example of non-referentially transparent,
    or \textit{referentially opaque}, function is
    \begin{equation}
        f(x) = \{x := x + 1;\ x\}
    \end{equation}
    This returns $x + 1$ but \textbf{also} changes the value of $x$ itself.
    Note that this implies that $x$ is not a value but a reference to a value;
    one can never change a value, it is only possible to change
    what one is referring to. (One can refer to a reference).

    This term originates from analytical philosophy
    and was introduced by Strachey.
    Referentially transparency implies that to find
    "the value of an expression which contains a sub-expression,
    the only thing we need to know about the sub-expression is its value.
    Any other features of the sub-expression,
    such as its internal structure,
    the number and nature of its components,
    the order in which they are evaluated or
    the colour of the ink in which they are written,
    are irrelevant to the value of the main expression"\cite{strachey2000fundamental}.\\
\end{definition}

\begin{definition}
    A \textit{pure} function is one which is both
    referentially transparent and which has the same semantics
    before and after the substitution of the function and its value
    has occurred. Pure functions are said to be free of effects.

    For example,
    a non-deterministic dice roll function is impure,
    it does not have the same semantics
    once a call of the function is replaced by a value.
\end{definition}

It should be noted that colloquially
the terms \textit{pure} and \textit{referentially transparent} are not fixed,
and are up to a point interchangeable when talking to some programmers.

Briefly consider \textit{coeffects} as a complication of this discussion.
Coeffects relate to the interaction of a program and its environment\cite{petricek2014coeffects},
(informally) consider it an inverse of a side-effect.
The program $ls$, which lists the files in the present working directory,
requires coeffects.
Haskell has an equivalent function $listDirectory$:
\begin{equation}
    listDirectory :: FilePath \rightarrow IO\ [FilePath]
\end{equation}

Obviously $ls$ is not referentially transparent,
it does not have any side-effects however.
It is definitely not a pure function.
Thus if we define pure functions to be those
which are free of effects then this
implies that coeffects are a subset of effects.
Coeffects, while interesting, are out of scope for this document,
but do serve to show the depth of the topic at hand.

Now we have established an intuition for what effects are,
we will explain why they lead to increased program complexity.

The first reason is that impure functions make it more
difficult to employ equational reasoning.
Purity is a powerful constraint to have on a function,
particularly when we want to prove properties of
our programs\cite{backus2007can}.

In a lazy language purity is even more consequential,
any pure function might be reordered or omitted.
If the function is not actually pure then this would
break the semantics of the program.

There exists a relation between referential transparency and
the (simply-typed) lambda calculus,
$\beta$-equality (or $\beta$-reduction) in particular.
\begin{equation}
\frac
{\Gamma, x : T \vdash t\prime : T\prime \quad \Gamma \vdash t : T}
{\Gamma \vdash (\lambda x.t\prime)t = t\prime[t/x]: T\prime }
\end{equation}
Consider $y, z : T$ where $y$ is a base value,
and $z$ is a function call ($z = \lambda x.t\prime$)
which returns the base value $y$,
\begin{align}
    z\,t &= (\lambda x.t\prime)t \\
         &= t\prime[t/x]
\end{align}
If $z$ is not pure then we do not have $\beta$-equality, i.e.
\begin{equation}
    z\,t \neq y
\end{equation}
$\beta$-equality is one of the fundamental properties
of the simply typed lambda calculus.
Abandoning purity means a departure from
the roots of functional programming.

The next reason why purity is desirable is that
a great deal of trouble can happen when the semantics
of a program is implicit rather than explicit.
Effects are usually not explicit in type systems,
this makes engineering and
maintaining larger programs
much more complicated.
Thus if untracked effects are a considerable source of errors then we should
wish to track them as best as possible.

Effect systems\cite{jouvelot1991algebraic} have been proposed as a solution,
"an effect system labels each function with its possible effects,
so a function type is now written $\tau \rightarrow \sigma \tau\prime$,
indicating a function that may have effects delimited by $\sigma$".
This method sees mainstream use in Java,
checked exceptions are an example of an effect system.
Wadler and Thiemann\cite{wadler2003marriage}
unify this approach with monads, which we will examine in this document.

The key benefit of type systems is that they allow us to identify erroneous programs
systematically,
we want to maximise the amount of erroneous programs the type-checker will not accept.
Being able to track and type effects is hugely beneficial.

Summarily, we can can not avoid effects so
we have to ask in what ways can we manage effects?
What tools do we have as a programmer at our disposal?
We will examine two ideas, both originating from category theory,
which provide us tools for effectful programming.
